% !TEX root = ../hmodel.tex
\section{Conclusion}
\label{sec:conclusion}

In this paper we have introduced the use of sphere-meshes as a novel geometric representation for articulated tracking. We have demonstrated how this representation yields excellent results for real-time registration of articulated geometry, and presented a calibration algorithm to estimate a per-user tracking template. We have validated our results by demonstrating qualitative as well as quantitative improvements over the state-of-the-art. Our volumetric model can be thought of as a generalization of the spherical models presented in~\cite{sridhar2015fast,qian2014realtime}, and the cylinder models of~\cite{oiko2011hand,tagliasacchi2015robust}. It is also related to the convex body model from~\cite{melax2013dynamics}, with the core advantage that its control skeleton compactly parameterizes its geometry. Our calibration optimization is related to the works in~\cite{taylor2014user,khamis15learning,tan2016fitsglove}, with a fundamental difference: the innate simplicity of sphere-meshes substantially simplifies the algorithmic complexity of calibration and tracking algorithms. This considered, we believe that with the use of \emph{compute shaders}, articulated tracking on the GPU can become as effortless and efficient as simple mesh rasterization.

\paragraph{Limitations and future work} 
% \Anastasia{The topology for a hand needs to be found only once and for all, and we have already found it and can provide it to anyone who wants. For an arbitrary tracking objection the topology can be using, for example, ``Sphere-Meshes'' system.}
The topology of our template has been defined in a manual trial-and-error process. A more suitable topology \todo{could be estimated} by optimization, possibly even adapting the topology for specific users; For example, the work in~\cite{thiery2016spheremesh} could be extended to space-time point clouds. Similarly, one could think of a variant of~\cite{newcombe2015dynfusion} where sphere-meshes are instantiated on-the-fly.
% 
% While our generative tracker performs very well, there are extreme situations, like when fingers becomes completely occluded, for which the use of a discriminative solution becomes unavoidable.
\begin{draft}
The use of more advanced re-initialization techniques than \cite{qian2014realtime}, like \cite{krupka2014discriminative} or \cite{oberweger2015feedback}, would be beneficial. Further, we believe an interesting venue for future work is how to elegantly integrate \todo{per-frame} estimates into generative trackers.
\end{draft}
% 
Model calibration is currently done in pre-processing. For certain consumer applications, it would be desirable to calibrate the model online during tracking, as recently proposed for face tracking systems~\cite{bouaziz2013online}. 
% 
\todo{Our sphere-mesh models are a first approximations to the implicit functions lying at the core of the recently proposed geometric skinning techniques~\cite{vaillant2013implicit,vaillant2014robust}. Therefore, we believe the calibration of \emph{sphere-meshes} to be the first step towards photorealistic real-time hand modeling and tracking.}
