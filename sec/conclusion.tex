% !TEX root = ../hmodel.tex
\section{Conclusion}
\label{sec:conclusion}
\AT{What's the difference between discussion and conclusion?}

\begin{DRAFT}
Lorem ipsum dolor sit amet, consectetur adipisicing elit, sed do eiusmod tempor incididunt ut labore et dolore magna aliqua. Ut enim ad minim veniam, quis nostrud exercitation ullamco laboris nisi ut aliquip ex ea commodo consequat. Duis aute irure dolor in reprehenderit in voluptate velit esse cillum dolore eu fugiat nulla pariatur. Excepteur sint occaecat cupidatat non proident, sunt in culpa qui officia deserunt mollit anim id est laborum.
Lorem ipsum dolor sit amet, consectetur adipisicing elit, sed do eiusmod tempor incididunt ut labore et dolore magna aliqua. Ut enim ad minim veniam, quis nostrud exercitation ullamco laboris nisi ut aliquip ex ea commodo consequat. \end{DRAFT}

\begin{figure}[t!]
\centering
\begin{overpic} 
[width=\linewidth]
% [width=\linewidth,grid,tics=10]
{\currfiledir/item.pdf}
\put(9,56){{\small $E_{2D}$ }}
\put(22,57.15){{\tiny \cite{tagliasacchi2015robust}}}
\put(22,55.15){{\tiny [Proposed Method]}}
%
\put(44.5,56){{\small $E_{3D}$ }}
\put(57,57.15){{\tiny \cite{tagliasacchi2015robust}}}
\put(57,55.15){{\tiny [Proposed Method]}}
% 
\put(11,2){{\small \emph{tayl1} }}
\put(22,2){{\small \emph{srid1} }}
\put(32,2){{\small \emph{srid2} }}
\put(43,2){{\small \emph{srid3} }}
\put(54,2){{\small \emph{srid4} }}
\put(64.5,2){{\small \emph{shar1} }}
\put(75,2){{\small \emph{shar2} }}
\put(86,2){{\small \emph{shar2} }}
% 
\end{overpic}
\caption{
% 
% 
Aggregated 2D/3D tracking performance metrics of the proposed method compared to \protect\cite{tagliasacchi2015robust}. 
% 
In the additional materials we report error plots through time for the aggregated data above.
% Our calibrated convolution model often helps the generative  technique from incurring in loss-of-tracking, these are visualized are spike in the errors in the
% 
% 
}
\label{fig:barchart}
\end{figure}

\paragraph{Limitations} 
The topology of our template has been defined in a manual trial-and-error process. A more suitable topology can probably be found using an optimization process, possibly even adapting the topology for specific users, which would be an interesting avenue for future research. $\quad$ Model calibration is currently done separately before the tracking starts. For certain consumer applications, it would be desirable to calibrate the model online during tracking, as recently proposed for face tracking systems~\cite{bouaziz2013online}. This, however, might be more challenging for hand tracking, where a significant amount of geometry is at grazing angles (e.g.\ finger silhouettes), which often leads to data culling and thus incomplete scans.

\paragraph{Future Work}
\AT{FUTURE: extension of our method to full body tracking?}

\AT{FUTURE: with the introduction of compute shaders, it will make complete sense to perform brute-force closest point lookup on the GPU. Just like today nobody uses CPU for rendering a 3D scene in real-time, it is possible nobody will perform alignment queries on the CPU. I think the first version of kinect was doing much of the computation on an FPGA...}

\AT{FUTURE: generate convolution models in a neural network? (given skeleton)}