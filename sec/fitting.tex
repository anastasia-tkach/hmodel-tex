% !TEX root = ../hmodel.tex

\section{Modeling}
\label{sec:modeling}

\brief{Multi-Pose Data}
Our calibration procedure tailors our template model to a specific user from a set of $N$ 3D measurements $\{ \depth_1 \dots \depth_N \}$ of the user's hand in different poses. Multiple measurements are necessary as it is not possible to understand the kinematic behavior by analyzing static geometry, while the redundancy of information improves fitting precision. Further, in monocular monocular acquisition this is essential, as single-view data is highly incomplete making the problem ill-posed. In our research we have experimented with datasets $\{\depth_i\}$ acquired via from multi-view stereo (e.g.\ \emph{Agisoft Photoscan~\textcopyright}), as well as a single RGBD sensor. Our calibration formulation can be employed for both acquisition modalities \todo{with minimal changes}.

\begin{figure}[t!]
\centering
\begin{overpic} 
[width=\linewidth]
% [width=\linewidth,grid,tics=10]
{fig/posing/item.pdf}
% \put(10,10){\todo{\Large Overlay Text}}
\end{overpic}
\caption{
% 
% 
(a) A visualization of the posed kinematic frames ${\bar{T}}_*$.
(b) The kinematic chain and number of degrees of freedom for posing our convolution model.
Tracking quality with (c) optimal and (d) non-optimal kinematic transformation frames. 
% 
% 
}
\label{fig:posing}
\end{figure}
\paragraph{Kinematics}
The rest-pose geometry of a convolution model is fully specified by two matrices specifying the set of sphere positions $\centers$ and radii $\radii$. The geometry is then posed through the application of  kinematic chain transformations; see \Figure{posing}\todo{a}. Given a point $\bar\point$ on the model $\model$ at rest pose, its 3D position after posing can be computed by evaluating the expression:
% 
\begin{equation}
\point = \left[ \Pi_{k \in K(i)} \mathbf{\bar{T}}_k \mathbf{T}_k \mathbf{\bar{T}}_k^{-1} \right] \bar\point
\label{eq:kinematic}
% we apply the posing transformation, then we re-apply the rest pose transformation to bring the point back in world coordinates
\end{equation}
%
where $\mathbf{T}_*$ are the \emph{pose} transformations parameterized by $\parpose$ and $\Pi$ left multiplies matrices by recursively traversing the kinematic chain $K(i)$ of element $i$ towards the root. Each node $k$ of the kinematic chain is associated with an orthogonal frame $\mathbf{\bar{T}}_k$ according to which local transformations are specified. In most tracking systems the frames $\mathbf{\bar{T}}_*$ are manually set by a 3D modeling artist and kept fixed across users. Incorrectly specified kinematic frames can be highly detrimental to tracking quality; see \Figure{posing}\todo{(b,c)} and \Video{00:00}. Therefore, in our formulation the kinematic structure is directly optimized from acquired data.

\paragraph{Formulation}
Let $\parpose_n$ be the \emph{pose} parameters optimally aligning the rest-pose template to the data frame $\depth_n$, and $\parposture$ be the \emph{posture} parameters representing the transformations $\mathbf{\bar{T}}_*$ via Euler angles. For notational brevity, we also define $\pars_n=[\parpose_n, \parposture, \centers, \radii]$. Our calibration optimization can then be written as:
% 
\begin{eqnarray}
\parposture, \centers, \radii = \argmin_{\{\pars_n\}}
\sum_{n=1}^N 
\sum_{\mathcal{T} \in \termscalib} 
w_\mathcal{T} E_\mathcal{T}(\depth_n, \pars_n)
\label{eq:calibration}
\end{eqnarray}
% 
We employ a set of energies $\termscalib$ to account for different requirements. On one hand we want a model that is a good fit to the data; on the other, we seek a non-degenerate convolution template that has been piecewise-rigidly posed. The following calibration energies $\termscalib$ encode these requirements:
% 
\begin{description}[labelsep=0em,labelwidth=.4in,labelindent=1cm,itemsep=-.6em]
\item[d2m] data to model Hausdorff distance (approx.)
\item[m2d] model to data Hausdorff distance (approx.)
\item[rigid] convolution elements are posed rigidly
\item[valid] convolution elements should not degenerate
\end{description}
% 
To make this calibration more approachable numerically, we rewrite \Eq{calibration} as an alternating optimization problem:
% 
\begin{eqnarray}
\restcenters, \radii, \posedcenters =
\argmin_{\posedcenters, \radii} 
\sum_{n=1}^N 
\sum_{\mathcal{T} \in \termscalib}
w_\mathcal{T} E_\mathcal{T}(\depth_n, \centers_n, \radii)
\label{eq:step1}
\\
\parposture, \posedcenters =
\argmin_{\parposes,\parposture} 
\sum_{n=1}^N 
\sum_{\mathcal{T} \in \termscalib}
w_\mathcal{T} E_\mathcal{T}(\centers_n, \pars_n) 
\end{eqnarray}
% 
Our first step adjusts radii and sphere centers, by allowing the convolution model to fit to the data without any kinematic constraint beyond rigidity, and returning as a side product a set of \emph{per-frame} posed centers $\posedcenters$. Our second step, takes the set $\posedcenters$ and projects it onto the manifold of kinematically plausible template deformations. This results in the optimization of the rotational components of rest-pose transformations $\mathbf{\bar{T}}$, as their translational components are simply derived from $\centers$.

\paragraph{Optimization}
The energies above are non-linear and non-convex, but can be optimized offline, as real-time tracking only necessitates a pre-calibrated model. For this reason, we conveniently employ the $lsqnonlin$ Matlab routine, and provide the gradients of our energies as well as an initialization point.
\AnastasiaComment{lsqnonlin does not provide the gradients and the optimization point. Anyone who used it will know this. lsqnonlin can compute numerical gradients, but our problem optimized like this will probably run for years instead of minutes.}
 The initialization of $\restcenters$ is performed \todo{automatically} by anisotropically scaling the vertices of a generic template to roughly fit the rest pose. The initial transformation frame rotations $\parposture$ are retrieved from the default template, while $\parposes$ are obtained by either aligning the scaled template to depth images, or by executing inverse kinematics on a few manually selected keypoints (multiview stereo).
 \AnastasiaComment{We should say that we use our tracking system with automatically scaled model to get the initial poses. Because the initial poses for calibration from sensor data are extremely close to the final ones. The model will not align to the from the rest pose, only tracking can do this.}

\begin{figure*}[h!]
\centering
\begin{overpic} 
[width=\linewidth]
% [width=\linewidth,grid,tics=10]
{fig/calibration/item.pdf}
% \put(10,10){\todo{\Large Overlay Text}}
\end{overpic}
\caption{
% 
% 
A visualization of a few iterations of our calibration optimization procedure; see \Video{00:00}. Each quadrant displays a data frame $\depth_n$, $n=1 \dots 4$. Within each quadrant we show three iterations of the optimization. The model being calibrated here is the one employed for real-time tracking in \Video{00:00}.
% 
% 
}
\label{fig:calibration}
\end{figure*}

\subsection{Energies}
Our fitting energies are analogous to the ones used in tracking. They approximate the symmetric Hausdorff distance, but they are evaluated on a collection rather than a single data frame:
% 
\begin{eqnarray}
E_{d2m} = |\depth_n|^{-1} \sum_{\point \in \depth_n} \| \point - \proj_{\model(\pars)}(\point)\|_2^1 \\
E_{m2d} = |\model(\pars)|^{-1} \sum_{\pixel \in \model(\pars)} \| \pixel - \proj_{\depth_n}(\pixel)\|_2^1
\end{eqnarray}
% 
Note that the projection operator $\proj_{\depth_n}$ changes according to the type of input data. If a multi-view acquisition system is used to acquire a complete point cloud, then the projection operator fetches the closest point to $\point$ in the point cloud of frame $\depth_n$. If $\depth_n$ is acquired through monocular acquisition, then $\proj_{\depth_n}$ computes the 2D projection to the image-space silhouette of the model.

\paragraph{Rigidity}
It is essential to estimate a template that \emph{jointly} fits the set of data frames $\{ \depth_n \}$. We achieve this task by requiring each posed model to be a piecewise-rigid articulation of our rest pose. This can be achieved by requiring each segment $\edge$ in our posed centers $\centers_n$ to share the same length as the corresponding segment $\bar\edge$ in its rest pose configuration:
% 
\begin{equation}
E_{\text{rigid}} = \sum_{\edge \in \centers_n} (\| \edge \| - \| \bar\edge \|)^2
\end{equation}
% 
Note that only a subset of the edges of our control skeleton, as illustrated in \Figure{topology}, are required to satisfy the rigidity condition.

\paragraph{Validity}
\AnastasiaComment{both this and previous paragraph start with \textbf{it is essential}}.
It is essential to prevent the calibration optimization from producing degenerate configurations. For example, a pill degenerates into a sphere when one of its balls is fully contained within the the volume of other. Analogously a wedge can degenerate into a pill, or even a sphere. We monitor validity by an indicator function $\chi(\ball_i)$ evaluating to one if $\ball_i$ is degenerate and zero otherwise, thus defining the following penalty function:
% 
\begin{equation}
E_{\text{valid}} = \sum_{\ballcenter_k \in \centers_k} \chi(\ballcenter_k) \| \ballcenter_k - \proj_{\model \setminus \ball_k}(\ballcenter_k) \|_2^2
\end{equation}
% 
We make a conservative choice and use $\chi(\ballcenter_i)$, which verifies whether $\ballcenter_i$ is inside $\model \setminus \ball_i$, the model obtaining by removing a vertex, as well as all its adjacent edges, from $\skeleton$. 
