\begin{figure}[t!]
\centering
\begin{overpic}
[width=\linewidth]
% [width=\linewidth,grid,tics=10]
{\currfiledir/item.pdf}
\put(1,30){{\small$\%$}}
\put(1,64){{\small$\%$}}
\put(93,37.5){{\small $E_{3D}$}}
\put(93,04.5){{\small $E_{2D}$}}
\end{overpic}
\caption{
% 
%
\todo{Each plot visualizes on the $y$ axis the portion of frames with a mean error metric below the value reported on the $x$ axis. We employ the \handyseq{teaser} sequence for this purpose. Curves closer to the top-left quadrant indicate better performance.}
% 
% Our method is quantitatively compared to the one of~\protect\cite{tagliasacchi2015robust} on the \handyseq{teaser} sequence.
% 
% 
}
\label{fig:comp2}
\end{figure}
\section{Calibration}
\label{sec:modeling}

\brief{Multi-Pose Data}
Our calibration procedure adapts our template model to a specific user from a set of $N$ 3D measurements $\{ \depth_1 \dots \depth_N \}$ of the user's hand in different poses. Multiple measurements are necessary, as it is not possible to understand the kinematic behavior by analyzing static geometry, and the redundancy of information improves fitting precision. Further, in  monocular acquisition this redundancy is essential, as single-view data is highly incomplete\todo{,} making the problem ill-posed. In our research we have experimented with datasets $\{\depth_n\}$ acquired via multi-view stereo (e.g. \emph{Agisoft Photoscan}), as well as a single RGBD sensor. 
Our calibration formulation can be employed for both acquisition modalities.
\begin{draft}
Dynamic reconstruction frameworks such as~\cite{newcombe2015dynfusion} or \cite{innmann2016volume} could also be used to generate a dynamic template mesh, over which sphere-mesh decimation could be executed~\cite{thiery2016spheremesh}. 
\todo{However, as no public implementation is currently available,} it is currently unclear how well these methods would cope with loop-closure for features as small as human fingers.
\end{draft}

\paragraph{Kinematics}
The rest-pose geometry of our model is fully specified by two matrices specifying the set of sphere positions $\restcenters$ and \todo{the set of} radii $\radii$. The geometry is then posed through the application of  kinematic chain transformations; see \Figure{posing}a. Given a point $\bar\point$ on the model $\model$ at rest pose, its 3D position after posing can be computed by evaluating the expression:
% 
\begin{equation}
\point = \left[ \Pi_{k \in K(\bar\point)} \mathbf{\bar{T}}_k \mathbf{T}_k \mathbf{\bar{T}}_k^{-1} \right] \bar\point
\label{eq:kinematic}
% we apply the posing transformation, then we re-apply the rest pose transformation to bring the point back in world coordinates
\end{equation}
%
where $\mathbf{T}_*$ are the \emph{pose} transformations parameterized by $\parpose$ and $\Pi$ left multiplies matrices by recursively traversing the kinematic chain $K$ of point $\bar\point$ towards the root~\cite{buss_04}. 
Each node $k$ of the kinematic chain is associated with an orthogonal frame $\mathbf{\bar{T}}_k$ according to which local transformations are specified. In most tracking systems, the frames $\mathbf{\bar{T}}_*$ are manually set by a 3D modeling artist and kept fixed across users. However, incorrectly specified kinematic frames can be highly detrimental to tracking quality; see \Figure{posing}(c,d) and \VideoKinematic{}. Therefore, in our formulation, the kinematic structure (i.e. the matrices $\mathbf{\bar{T}}_*$) is directly optimized from acquired data.

\begin{figure}[t!]
\centering
\begin{overpic} 
[width=\linewidth]
% [width=\linewidth,grid,tics=10]
{\currfiledir/item.pdf}
\put(93,37.5){{\small $E_{3D}$}}
\put(93,04.5){{\small $E_{2D}$}}
\put(1,30){{\small $\%$}}
\put(1,64){{\small $\%$}}
\end{overpic}
\caption{\todo{Calibrating progressively improves the 2D/3D tracking metrics, showing a remarkable improvement in tracking fidelity from~\protect\cite{tagliasacchi2015robust} to~\citeme{}}}
\label{fig:calibeval}
\end{figure}
\begin{figure*}[h!]
\centering
\begin{overpic} 
[width=\linewidth]
% [width=\linewidth,grid,tics=10]
{fig/calibration/item.pdf}
% \put(10,10){\todo{\Large Overlay Text}}
\end{overpic}
\caption{
% 
% 
A visualization of a few iterations of our calibration optimization procedure; see \Video{00:00}. Each quadrant displays a data frame $\depth_n$, $n=1 \dots 4$. Within each quadrant we show three iterations of the optimization. The model being calibrated here is the one employed for real-time tracking in \Video{00:00}.
% 
% 
}
\label{fig:calibration}
\end{figure*}

\newpage
\paragraph{Formulation}
Let $\parpose_n$ be the \emph{pose} parameters optimally aligning the rest-pose template to the data frame $\depth_n$, and $\parposture$ be the \emph{posture} parameters representing the transformations $\mathbf{\bar{T}}_*$ via Euler angles. For notational brevity, we also define $\pars_n=[\parpose_n, \parposture, \restcenters, \radii]$. Our calibration optimization can then be written as:
% 
\begin{eqnarray}
% \parposture, \centers, \radii =
\argmin_{\{\pars_n\}}
\sum_{n=1}^N 
\sum_{\mathcal{T} \in \termscalib} 
w_\mathcal{T} E_\mathcal{T}(\depth_n, \pars_n)
\label{eq:calibration}
\end{eqnarray}
% 
We employ a set of energies $\termscalib$ to account for different requirements. On one hand we want a model that is a good fit to the data; on the other, we seek a non-degenerate convolution template that has been piecewise-rigidly posed. The following calibration energies $\termscalib$ encode these requirements:
% 
\begin{description}[labelsep=0em,labelwidth=.4in,labelindent=1cm,itemsep=-.6em]
\item[d2m] data to model distance
\item[m2d] model to data distance
\item[rigid] elements are posed rigidly
\item[valid] elements should not degenerate
\end{description}
% 
To make this calibration more approachable numerically, we rewrite \Eq{calibration} as an alternating optimization problem:
% 
\begin{eqnarray}
% \restcenters, \radii, \posedcenters =
\argmin_{\posedcenters, \restcenters, \radii} &
\sum_{n=1}^N 
\sum_{\mathcal{T} \in \termscalib}
w_\mathcal{T} E_\mathcal{T}(\depth_n, \centers_n, \restcenters, \radii)
\label{eq:step1}
\\
% \parposes, \parposture =
\argmin_{\parposes,\parposture} &
\sum_{n=1}^N 
\sum_{\mathcal{T} \in \termscalib}
w_\mathcal{T} E_\mathcal{T}(\centers_n, \pars_n) 
\end{eqnarray}
% 
\todo{Our first step adjusts rest-pose sphere centers $\restcenters$ and radii $\radii$,} by allowing the convolution model to fit to the data without any kinematic constraint beyond rigidity, and returning as a side product a set of \emph{per-frame} posed centers $\posedcenters$. 
Our second step takes the set $\posedcenters$ and projects it onto the manifold of kinematically plausible template deformations. 
This results in the optimization of the rotational components of rest-pose transformations $\mathbf{\bar{T}}_*$, as their translational components are simply derived from $\restcenters$.

\paragraph{Optimization}
The energies above are non-linear and non-convex, but can be optimized offline, as real-time tracking only necessitates a pre-calibrated model. For this reason, we conveniently employ the $lsqnonlin$ Matlab routine, which requires the gradients of our energies as well as an initialization point.
The initialization of $\restcenters$ is performed automatically by anisotropically scaling the vertices of a generic template to roughly fit the rest pose. The initial transformation frame rotations $\parposture$ are retrieved from the default template, while $\parposes$ are obtained by either aligning the scaled template to depth images, or by executing inverse kinematics on a few manually selected keypoints (multi-view stereo).
% 
% \AnastasiaComment{We should say that we use our tracking system with automatically scaled model to get the initial poses. Because the initial poses for calibration from sensor data are extremely close to the final ones. The model will not align to the from the rest pose, only tracking can do this. AT: I already wrote that, the sentence ``aligning the scaled template to the depth images''}
% 
Our (unoptimized) Matlab script calibrates the model within a few minutes for all our examples.


\subsection{Energies}
Our fitting energies are analogous to the ones used in tracking. They approximate the symmetric Hausdorff distance, but they are evaluated on a collection rather than a single data frame:
% 
\begin{eqnarray}
E_{d2m} = |\depth_n|^{-1} \sum_{\point \in \depth_n} \| \point - \proj_{\model(\pars)}(\point)\|_2^1 \\
E_{m2d} = |\model(\pars)|^{-1} \sum_{\pixel \in \model(\pars)} \| \pixel - \proj_{\depth_n}(\pixel)\|_2^1
\end{eqnarray}
% 
Note that the projection operator $\proj_{\depth_n}$ changes according to the type of input data. If a multi-view acquisition system is used to acquire a complete point cloud, then the projection operator fetches the closest point to $\point$ in the point cloud of frame $\depth_n$. If $\depth_n$ is acquired through monocular acquisition, then $\proj_{\depth_n}$ computes the 2D projection to the image-space silhouette of the model.

\paragraph{Rigidity}
It is essential to estimate a \todo{single user template that, once articulated,} \emph{jointly} fits the set of data frames $\{ \depth_n \}$. For this purpose we require each posed model to be a piecewise-rigid articulation of our rest pose. \todo{This can be achieved by constraining each segment $\{ (\ballcenter_{n,i}, \ballcenter_{n,j})\:|\: ij \in \skeleton \}$ of $\centers_n$ to have the same length as the corresponding segment $(\bar\ballcenter_i, \bar\ballcenter_j)$ of the rest pose configuration $\restcenters$}:
% 
\begin{equation}
E_{\text{rigid}} = 
% \sum_{\ballcenter_{n,*} \in \centers_n}
\sum_{ij \in \skeleton} (\| \ballcenter_{n,i} - \ballcenter_{n,j} \| - \| \bar\ballcenter_i - \bar\ballcenter_j \|)^2
\end{equation}
% 
Note that only a subset of the edges of our control skeleton, as illustrated in \Figure{topology}, are required to satisfy this rigidity condition.

% \newpage
\paragraph{Validity}
The calibration optimization should avoid producing degenerate configurations \todo{in our \emph{rest pose} template $\restcenters$}. For example, a pill degenerates into a sphere when one of its balls is fully contained within  the volume of the other. Analogously, a wedge can degenerate into a pill or a sphere. We monitor validity by an indicator function $\chi(\bar\ball_i)$ that evaluates to one if $\bar\ball_i$ is degenerate and zero otherwise.
% 
\todo{We make a conservative choice and use $\chi(\bar\ball_i)$, which verifies whether $\bar\ballcenter_i$ is inside $\bar\oneelement \setminus \bar\ball_i$, the element obtained by removing a vertex, as well as all its adjacent edges, from $\bar\oneelement$.}
% 
This leads to the following \todo{conditional} penalty function:
% 
\begin{equation}
E_{\text{valid}} = 
\sum_{\bar\oneelement \in \restcenters}
\sum_{\bar\ball_i \in \bar\oneelement} 
\chi(\bar\ball_i) 
\| \bar\ballcenter_i - \proj_{\bar\oneelement \setminus \bar\ball_i}(\bar\ballcenter_i) \|_2^2
\end{equation}
% 
 
