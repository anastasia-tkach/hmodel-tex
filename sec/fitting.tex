% !TEX root = ../hmodel.tex

\section{Modeling}
\label{sec:modeling}

\brief{Multi-Pose Data}
Our calibration procedure tailors our template model to a specific user from a set of $N$ 3D measurements $\{ \depth_1 \dots \depth_N \}$ of the user's hand in different poses. Multiple measurements are necessary as it is not possible to understand the kinematic behavior by analyzing static geometry, while the redundancy of information improves fitting precision. Further, in monocular monocular acquisition this is essential, as single-view data is highly incomplete making the problem ill-posed. In our research we have experimented with datasets $\{\depth_i\}$ acquired via from multi-view stereo (e.g.\ \emph{Agisoft Photoscan~\textcopyright}), as well as a single RGBD sensor. Our calibration formulation can be employed for both acquisition modalities with minimal \todo{difference}.

\begin{figure}[t!]
\centering
\begin{overpic} 
[width=\linewidth]
% [width=\linewidth,grid,tics=10]
{fig/posing/item.pdf}
% \put(10,10){\todo{\Large Overlay Text}}
\end{overpic}
\caption{
% 
% 
(a) A visualization of the posed kinematic frames ${\bar{T}}_*$.
(b) The kinematic chain and number of degrees of freedom for posing our convolution model.
Tracking quality with (c) optimal and (d) non-optimal kinematic transformation frames. 
% 
% 
}
\label{fig:posing}
\end{figure}
\paragraph{Kinematics}
The rest-pose geometry of a convolution model is fully specified by two matrices specifying the set of sphere positions $\centers$ and radii $\radii$. The geometry is then posed through the application of  kinematic chain transformations; see \Figure{posing}\todo{a}. Given a point $\bar\point$ on the model $\model$ at rest pose, its 3D position after posing can be computed by evaluating the expression:
% 
\begin{equation}
\point = \left[ \Pi_{k \in K(i)} \mathbf{\bar{T}}_k \mathbf{T}_k \mathbf{\bar{T}}_k^{-1} \right] \bar\point
\label{eq:kinematic}
% we apply the posing transformation, then we re-apply the rest pose transformation to bring the point back in world coordinates
\end{equation}
%
where $\mathbf{T}_*$ are the \emph{pose} transformations parameterized by $\parpose$ and $\Pi$ left multiplies matrices by recursively traversing the kinematic chain $K(i)$ of element $i$ towards the root. Each node $k$ of the kinematic chain is associated with an orthogonal frame $\mathbf{\bar{T}}_k$ according to which local transformations are specified. In most tracking systems the frames $\mathbf{\bar{T}}_*$ are manually set by a 3D modeling artist and kept fixed across users. Incorrectly specified kinematic frames can be highly detrimental to tracking quality; see \Figure{posing}\todo{(b,c)} and \Video{00:00}. Therefore, in our formulation the kinematic structure is directly optimized for every user from acquired data.

\paragraph{Formulation}
Let $\parpose_n$ be the \emph{pose} parameters optimally aligning the rest-pose template to the data frame $\depth_n$, and $\parposture$ be the \emph{posture} parameters representing the transformations $\mathbf{\bar{T}}_*$ via Euler angles. For notational brevity, we also define $\pars_n=[\parpose_n, \parposture, \centers, \radii]$. Our calibration optimization can then be written as:
% 
\begin{eqnarray}
\parposture, \centers, \radii = \argmin_{\{\pars_n\}}
\sum_{n=1}^N 
\sum_{\mathcal{T} \in \termscalib} 
w_\mathcal{T} E_\mathcal{T}(\depth_n, \pars_n)
\label{eq:calibration}
\end{eqnarray}
% 
We employ a set of energies $\termscalib$ to account for different requirements. On one hand we want a model that is a good fit to the data; on the other, we seek a non-degenerate convolution template that has been piecewise-rigid posed. The following calibration energies $\termscalib$ encode these requirements:
% 
\begin{description}[labelsep=0em,labelwidth=.4in,labelindent=1cm]
\item[d2m] data to model Hausdorff distance (approx)
\item[m2d] model to data Hausdorff distance (approx)
\item[rigid] convolution elements are posed rigidly
\item[valid] convolution elements should not degenerate
\end{description}
% 
To make this calibration more approachable numerically, we rewrite \Eq{calibration} as an alternating optimization problem:
% 
\begin{eqnarray}
\posedcenters, \radii =
\argmin_{\posedcenters, \radii} 
\sum_{n=1}^N 
\sum_{\mathcal{T} \in \termscalib}
w_\mathcal{T} E_\mathcal{T}(\depth_n, \centers, \radii)
\label{eq:step1}
\\
\restcenters =
\argmin_{\restcenters,\parposes,\parposture} 
\sum_{n=1}^N 
\sum_{\mathcal{T} \in \termscalib}
w_\mathcal{T} E_\mathcal{T}(\posedcenters, \pars_n) 
\end{eqnarray}
% 
Our first step adjusts radii and sphere centers, by allowing the convolution model to fit to the data without any kinematic constraint beyond rigidity, and returning $\posedcenters$: a set of \emph{per-frame} posed centers.
The second step, takes the set $\posedcenters$ and returns a 


pose and posture are optimized jointly by fixing the geometry of the template and aligning multiple instances to each frame. The kinematic chain is adjusted by refining the rest-pose transformations $\mathbf{\bar{T}}_k$; note only the rotational degrees of freedom of $\mathbf{\bar{T}}_k$ are optimized for, as frame translations are inferred from $\skeleton$. 
\todo{This optimization projects the posed centers $\posedcenters$ onto the closest point on the manifold of convolution surfaces respecting the given kinematic constraints.}

\paragraph{Optimization and Initialization}
The energies above are non-linear and non-convex, but can be optimized offline, as in real-time tracking we can use a pre-calibrated model. For this reason, we can use the convenient $lsqnonlin$ routine from Matlab, requiring us to specify the energies gradients (see \Appendix{gradients}) as well as an initialization point. The initialization of $\restcenters$ is performed \todo{automatically} by anisotropically scaling the vertices of a generic template to roughly fit the rest pose. The initial transformation frame rotations $\parposture$ are retrieved from the default template, while $\parposes$ are obtained by either aligning the rough template to the input data (for depth images) or by executing IK registration of a few manually selected key-points (for MVS reconstructed models).

\begin{figure*}[h!]
\centering
\begin{overpic} 
[width=\linewidth]
% [width=\linewidth,grid,tics=10]
{fig/calibration/item.pdf}
% \put(10,10){\todo{\Large Overlay Text}}
\end{overpic}
\caption{
% 
% 
A visualization of a few iterations of our calibration optimization procedure; see \Video{00:00}. Each quadrant displays a data frame $\depth_n$, $n=1 \dots 4$. Within each quadrant we show three iterations of the optimization. The model being calibrated here is the one employed for real-time tracking in \Video{00:00}.
% 
% 
}
\label{fig:calibration}
\end{figure*}

\subsection{Energies}
We now describe in detail the various energies of our optimization.

\TODO{RE-WRITE THE FITTING ENERGIES?}
\begin{DRAFT}
\begin{equation}
E_{\model \rightarrow \depth} = \sum_{\point \in \model(\pars_n)} \| \point - \proj_{\depth_n}(\point)\|_2^1
\label{eq:m2d}
\end{equation}
As we are seeking for a model which is strictly sufficient to represent our data, we need to ensure that \todo{no portion of the model remains unused} \AT{better way of saying this?}. 
The data-to-model term encodes an approximation of the \emph{one-sided} Hausdorff distance, but this is insufficient in applications where the tracking template is unknown. An approximation of the \emph{symmetric} Hausdorff distance can be obtained by also accounting for model-to-data correspondences
% 
Note that the projection operator $\proj_{\depth_n}$ changes according to the type of input data being used. If a multi-view acquisition system is used to acquire a complete point cloud, then the projection operator is similar to the one used for \Eq{d2m}: we build a kd-tree for the point cloud in frame $\depth_n$ and fetch the closest point to $\point$. Conversely, if the data in $\depth_n$ has been obtained through monocular acquisition, then $\proj_{\depth_n}$ computes the 2D projection to the image-space silhouette of the model.
\end{DRAFT}

\paragraph{Rigidity}
It is essential that we recover a template that \emph{jointly} fits the set of data frames $\{ \depth_n \}$. We achieve this task by requiring each posed model to be a piecewise-rigid articulation of our rest pose. This can be achieved by simply requiring that each segment $\edge$ in our posed control skeleton $\skeleton_n$ to have the same length as the corresponding segment $\bar\edge$ in the rest pose configuration:
% 
\begin{equation}
E_{\text{rigid}} = \sum_{\edge \in \skeleton_n} (\| \edge \| - \| \bar\edge \|)^2
\end{equation}
% 
Note that only a subset of the edges of our control skeleton, as illustrated in \Figure{topology}, are required to satisfy the rigidity condition.

\paragraph{Validity}
It is essential to prevent the calibration optimization from producing degenerate configurations. For example, a pill degenerates into a sphere when one of its balls is fully contained within the the volume of other. Analogously a wedge can degenerate into a pill, or even a sphere. We monitor validity by an indicator function $\chi(\ball_i)$ evaluating to one if $\ball_i$ is degenerate and zero otherwise.
Degeneracies can be avoided by including the energy term:
% 
\begin{equation}
E_{\text{valid}} = \sum_{\ballcenter_k \in \skeleton} \chi(\ballcenter_k) \| \ballcenter_k - \proj_{\model \setminus \ball_k}(\ballcenter_k) \|_2^2
\end{equation}
% 
In this equation, we make a conservative choice and use $\chi(\ballcenter_i)$, which verifies whether $\ballcenter_i$ is inside $\model \setminus \ball_i$, the model obtaining by removing a vertex, as well as all its adjacent edges, from $\skeleton$. 

\endinput

\paragraph{Rigidity - OLD}
\begin{DRAFT}
We encode this requirement by requiring each edge $\edge_k$ in each posed skeleton $\skeleton_n$ to be a rotation of rotation $R_{k}$ of its rest-pose configuration~$\bar\edge_k$: 
\begin{equation}
E_{\text{rigid}} = \sum_{\edge_k \in \skeleton_n} \| \edge_k  - R_{k}{\bar\edge_k} \|_2^2
\label{eq:rigidity}
\end{equation} 
Note that only a subset of the edges of our control skeleton, as illustrated in \Figure{topology}, are required to satisfy this rigidity requirement. At the cost of introducing auxiliary variables $R_{i}$, this energy becomes independent from pose $\parposes$ and posture $\parposture$ angles.
\end{DRAFT}

