% Problem Statement: What's the problem you want to solve?
% 
% Motivation: Why is this an interesting problem? Who cares about it? Why now? Why is it appropriate for the conference audience?
% 
% Research Gap, Novelty: Why is new research required? Why can the problem not be solved with existing methods? How does the proposed solution differ from and/or improve upon existing work?
% 
% Technical Contribution: What's the key technical idea to solve the problem? Why is it beautiful?
% 
% Applications / Future Work: What will your solution enable? How does it project into the future? How will it inspire future work?



% NOTE: the paragraph names can be removed later on
\section{Introduction}

%--- Why hand-tracking is fundamental (something abour AR/VR?)
% Hand tracking is a process of accurately reconstructing shape and articulation of human hands. It is a crucial component of natural human-computer interfaces and animation of humanoid avatars.
\TODO{In our world, our hands are our main mean of interaction, and with the advent of augmented and virtual reality experiences, there is a compelling need ...}
%--- Camera-based tracking & Physics
Accurate real-time hand tracking is therefore a fundamental technical challenge \todo{In AR and VR applications a 3D hand model can properly interact with 3D objects, establish a realistic contact and disappear behind them. Given that, the degree of immersion into virtual reality depends on whether a user sees own realistic hands (find a study mentioned by Leap Motion).} \todo{Lorem ipsum dolor sit amet, consectetur adipisicing elit, sed do eiusmod tempor incididunt ut labore et dolore magna aliqua. Ut enim ad minim veniam, quis nostrud exercitation ullamco laboris nisi ut aliquip ex ea commodo consequat. Lorem ipsum dolor sit amet, consectetur adipisicing elit, sed do eiusmod tempor incididunt ut labore et dolore magna aliqua. Ut enim ad minim veniam, quis nostrud exercitation ullamco laboris nisi ut aliquip ex ea commodo consequat.}

%--- Combination of model+appearance
\paragraph{Tracker: discriminative v.s. generative.}
Modern systems for real-time tracking \cite{sridhar2015fast,sharp2015accurate},  rely on a combination of \emph{discriminative} approaches like \cite{oberweger2015feedback}, and \emph{generative} approaches such as \cite{tagliasacchi2015robust}. The per-frame re-initialization of discriminative methods prevents error propagation by offering a continous recovery from tracking failure. As these discriminative models are learnt from data, they typically only estimate a coarse and jittery pose. Therefore, generative models are used to refine the estimate by aligning a geometric template of the user hand to the measured point cloud \AT{temporal?}. It is not surprising that the quality of the template directly affects the quality of pose refinement; see \Figure{teaser}. 
\TODO{mention that no need to re-init at high FPS?}.
% 
Therefore, the main goal of this paper is to explore novel tracking templates that strike an optimal \emph{balance} between accuracy and performance, that is, a model that is able to more accurately capture the user's geometry, while retaining the ability to answer registration queries in close form with very high efficiency.

\paragraph{Surface and volumetric templates.}
In modern digital production representing objects by a piecewise linear meshing of their surface (i.e. triangular or quad meshes) is the de-facto standard. However, unlike volumetric models~\cite{bloomenthal1997book}, surface representations cannot efficiently answer queries such as the distance from the point to the object's boundary, or whether a point lies inside/outside the model~\cite[Ch.1]{botsch2010book}. In tracking applications these queries play a fundamental role, as the optimization attempts to find configurations where the average \emph{distance} from model to data is minimized. Similarly, a tracker should prevent the model from assuming implausible configurations, for example by preventing self-intersections as measured by inside/outside predicates. For all these reasons, volumetric models appear optimal for registration applications; indeed, recent compelling results in joint rigid registration and reconstruction~\cite{newcombe2011kinfu} as well as its recent non-rigid variant~\cite{newcombe2015dynfusion} leverage volumetric models. One important observations is that such techniques assume the frame-rate is high compared to camera and user motion, a condition that is surely not realizable in our setting. \AT{something about the fact that deforming a surface is easier than deforming a volume?} To tackle this challenge, in this paper we propose to employ a \emph{hybrid} model for tracking that inherits the advantages of surface and volumetric representations. 

\newpage
\begin{figure}[t!]
\begin{overpic} 
[width=\linewidth]
% [width=\linewidth,grid,tics=5]
{fig/convsurf/item.pdf}
\put(20,51){\small{$\mathcal{M}$}}
\put(70,51){\small{$\mathcal{M}$}}
\end{overpic}
\vspace{-.3in}
\caption{
% 
% 
The mesh {\small$\mathcal{M}$} explicitly identifies sphere positions and controls the union convolution operator to generate an implicit function. The zero-crossing of this implicit function describes the convex-hull of our spheres. \AT{redo this figure as I stole if from [ryoichi-sig13]!! \TODO{add the vertices names $m_1$ and $m_2$}}
% 
% \ignore{~\cite{ryoichi_sig13}}.
}
\label{fig:convsurf}
\end{figure}

\paragraph{Hybrid tracking model.}
The model we propose in this paper is a variant of a convolution surface~\cite{bloomenthal1991convolution}, and its fundamental building block is illustrated in \Figure{convsurf}. Such a construct is nothing but the zero iso-surface of the scalar function:
\begin{equation}
\phi(\mathbf{x}) = \min \int_{\mathbf{c} \in \mathcal{M}} \mathcal{B}_{\mathbf{c}, r(\mathbf{c})}(\mathbf{x}) \: d\mathbf{c},
\label{eq:convsurf}
\end{equation}
where $\mathcal{M}$ is a skeletal control mesh (a segment or a triangle with respect to the simple example in \Figure{convsurf}), and $\mathcal{B}$ is the implicit function of a sphere parameterized by its center $\mathbf{c}$ and radii $r$:
\begin{equation}
\mathcal{B}_{\mathbf{c}, r(\mathbf{c})}(\mathbf{x}) = \|\mathbf{x}-\mathbf{c}\|^2 - r(\mathbf{c})^2.
\end{equation}
The spheres centers $\mathbf{c}$ span the skeleton $\mathcal{M}$, while the radii is a function of the position $\mathbf{c}$ within an element, and more specifically, it is linearly interpolated from values specified on the skeletal mesh vertices $r(m_*)$. This is indeed a \emph{hybrid} model, as \Eq{convsurf} defines an implicit surface $\mathcal{S} = \{\mathbf{x} \in \mathbb{R}^n | \phi(\mathbf{x})=0 \}$, while the underlying skeleton $\mathcal{M}$ is an explicit representation (a piecewise-parameterized \todo{surface}). We generalize this basic construct to devise a model suitable to represent a human hand; see \Figure{topology}.
\begin{figure}[t!]
\centering
\begin{overpic} 
[width=\linewidth]
% [width=\linewidth,grid,tics=10]
{fig/topology/item.pdf}
%{fig/topology/topology}
% \put(10,10){\todo{fig:topology}}
\end{overpic}
\caption{
% 
% 
(left) The skeleton $\skeleton$ parametrizes the convolution surface providing a radii value property on vertices. 
In our template, articulated components are shown in {\color{darkgreen} dark green} while flexible components in {\color{purple}purple}.
% 
(right) Calibration instantiates our template by adjusting the skeletal vertex positions and radii. 
% 
% 
}
\label{fig:topology}
\end{figure}

\paragraph{Tracking and calibration with convolution models.}
Our novel tracking model has two significant advantages. (1) First of all, distance queries to $\surface$ can be executed by measuring the distance to the skeletal structure $\skeleton$. The number of elements in $\skeleton$ is significantly smaller (\todo{25} in our model) than the number of polygons in a typical triangular mesh representation of $\surface$~\cite{thiery2013sphere}. Therefore, not only our distance queries can be executed with great efficiently as we can evaluate distance queries to $\skeleton$ in brute force but this leads to an algorithm that is trivially parallelizable and that executes with a fixed framerate. (2) Our hand model parameterization is also compact, as we can generate a family of models by simply adjusting \emph{positions} and \emph{radii} of the control skeleton vertices $m_* \in \skeleton$. A direct approach to building a calibrated model is to let the artist manually adjust these quantities; this process is known as  \emph{ZSphere~{\textcopyright}} modeling in the popular \emph{ZBrush~\textcopyright} 3D modeling software.

\TODO{--- STOPPED HERE -- BACK TOMORROW ---}

\begin{figure}[b]
\centering
\begin{overpic} 
[width=\linewidth]
% [width=\linewidth,grid,tics=10]
{fig/coarsemodel/item.pdf}
\put(10,10){{\Large TODO}}
\put(40,10){{\Large TODO}}
\put(75,10){{\Large TODO}}
\end{overpic}
\vspace{-.25in}
\caption{(left) Tracking when the model from \protect\cite{tagliasacchi2015robust} is used without proper coarse scale calibration. (middle) A roughly manually calibrated model can help increasing the fitting fidelity, but tuning becomes increasingly difficult with more degrees of freedom. (right) The proposed automatically calibrated model.}
% \caption{(left) coarse hand model from \protect\cite{tagliasacchi2015robust} \todo{(right) hand model animated with linear blend skinning from \protect\cite{sharp2015accurate}}}
\label{fig:coarsemodel}
\end{figure}
\newpage

% While the integral in \Eq{convsurf} might be difficult to evaluate for generic skeletal structures $\mathcal{M}$,

% \TODO{why is it necessary, plus previous works}
\paragraph{Model calibration.}
As generative models perform tracking by fitting a geometric model to what is measured by the sensor, our template should be able to represent well the observed data. However, even in modern trackers, the discrepancy between the optimal model pose given the data and the true hand pose can be significant. For example, the sources of discrepancy in \Figure{coarsemodel} include the incorrect length of fingers and the lack of the \todo{[pinky-ring]} degree of freedom. \todo{In the literature, the creation of user-specific models for tracking is referred to as \emph{calibration}.}

\paragraph{LBS Looks Crappy}
\todo{The hand skinning quality is obviously important for digital avatars applications. The simple skinning approaches like linear blend skinning may generate implausible results.}

\begin{figure}[t!]
\centering
\begin{overpic} 
[width=\linewidth]
% [width=\linewidth,grid,tics=10]
{fig/handmodels/hand_model_representations.png}
% \put(10,10){\todo{\Large Overlay Text}}
\end{overpic}
\caption{Hand model representations from works of (a) Qian et. al., (b)  Oikonomidis et. al., (c) Melax et. al.  and (d) Sharp et. al. \AT{inline the names}}
\label{fig:onecol}
\end{figure}

% \begin{figure}[h!]
% \centering
% \hspace{-2em}
% \includegraphics[width=0.5\textwidth]{fig/hand_model_representations}
% \caption {Hand model representations from works of (a) Qian et. al., (b)  Oikonomidis et. al., (c) Melax et. al.  and (d) Sharp et. al}
% \label{fig:hand_model_representations}
% \end{figure}

\begin{figure}[t!]
\centering
\begin{overpic} 
[width=\linewidth]
% [width=\linewidth,grid,tics=10]
{fig/topology/item.pdf}
%{fig/topology/topology}
% \put(10,10){\todo{fig:topology}}
\end{overpic}
\caption{
% 
% 
(left) The skeleton $\skeleton$ parametrizes the convolution surface providing a radii value property on vertices. 
In our template, articulated components are shown in {\color{darkgreen} dark green} while flexible components in {\color{purple}purple}.
% 
(right) Calibration instantiates our template by adjusting the skeletal vertex positions and radii. 
% 
% 
}
\label{fig:topology}
\end{figure}


\begin{table}[!ht] 
	\centering
	\begin{tabular}{|p{2.5cm}|p{2.5cm}|p{2.5cm}|}
	\hline
 	& Hand pose  & Hand shape  \\
	\hline
	Triangular mesh with embedded skeleton, \cite{taylor2014user} & Vertices and bones positions & Vertices and bones positions	 \\
	\hline
	Cylinder model, \cite{tagliasacchi2015robust} & Cylinders size and transformations & Cylinders transformations	 \\
	\hline
	Convolution surfaces model & Positions and radii of control points & Positions of control points \\
	\hline
	\end{tabular}
	\vspace{1em}
	\caption{Comparison of different hand model representations}
	\label{table:representation_dependent_components}
\end{table}

\subsection{Convolution surfaces for model fitting}
The spheres and mixed cylinders/spheres hand model representations (Figure \ref{fig:hand_model_representations} a, b) are ubiquitous in hand tracking, because they are well suited for tracking tack per se (see next) and can be quickly to created manually. If a small number of  building blocks is used, the precision of the model is low, especially in the palm region. A higher precision can be obtained by increasing the number of primitives, which defeats the purpose of model simplicity. Convolution surfaces representation gives higher precision for the same number of building blocks. \textcolor{mygray}{Add experimental or theoretical support for convolution surfaces.}
The triangles mesh representation can approximate the hand to a high precision. However, it is not trivial to specify the model parts that are rigid and should be kept the same shape between the poses. This results in overfitting to local skin deformations. 

\subsection{Convolution surfaces for hand tracking}
For model based tracking the main operation is to find the closest point on the model for a given data point. This operation is can be done in closed for each rigid segment with spheres/cylinder and convolution surfaces model representation. 
For a triangular mesh this operation has complexity linear in number of triangles. Thus, it is necessary to simplifying assumptions and/or more complex optimization to allow the hand tracking system to run in real time. (Look how different systems deal with this problem). Moreover, the triangular mesh has (much) more degrees of freedom than the underlying problem. Without additional regularization, rigid parts of the hand model can deform to fit the data and the individual vertices can shift to fit the sensor noise.

\subsection{Convolution surfaces for hand skinning}
The Linear Blend Skinning approach used to pose the triangular mesh model in previous works (\cite{sharp2015accurate}, \cite{schroder2013analysis} ) creates artifacts, the fingers looks like made from rubber. The spheres/cylinders model is not suitable for realistic animation, therefore a re-targeting step to a template mesh is required. Retargeting does not only demand additional effort, but also brings additional imprecision. The state of the art approaches in hand skinning are implicit surfaces-based (\cite{vaillant2013implicit},  \cite{vaillant2014robust} ).  A convolution surfaces model serves a ready to use input for such an approach.

\subsection{Contributions}

\subsubsection*{1. Suggested a hand model representation well suited for user-specific adaption, tracking and animation}

TODO: 

\begin{itemize}
\item Modeling. Implement initialization for the model fitting by running Htrack and putting the hand in a similar pose. 
\item Skinning. Display the texture on the model. For that a parametrization is required. Potentially create ``bumps map'' to apply the errors from model fitting stage. 
\end{itemize}

\subsubsection*{1a. Proposed a single optimization framework for modeling and tracking}

TODO:

\begin{itemize}
\item Run a couple of iteration of ARAP for palm after IK optimization just with data term.
\end{itemize}

\subsubsection*{1b. Developed a tracking model that represents all the degrees of freedom of human hand without creating inexistent degrees. (Not guaranteed by the other approaches, look how model adaptation paper finds the degrees of freedom)}

\subsubsection*{Improved accuracy of state-of-art hand tracking system}

TODO: 

\begin{itemize}
\item Add outline term, time interpolation of initial joint angles, downsampling of the sensor data, fingertips term.
\end{itemize}