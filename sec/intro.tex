% Problem Statement: What's the problem you want to solve?
% 
% Motivation: Why is this an interesting problem? Who cares about it? Why now? Why is it appropriate for the conference audience?
% 
% Research Gap, Novelty: Why is new research required? Why can the problem not be solved with existing methods? How does the proposed solution differ from and/or improve upon existing work?
% 
% Technical Contribution: What's the key technical idea to solve the problem? Why is it beautiful?
% 
% Applications / Future Work: What will your solution enable? How does it project into the future? How will it inspire future work?



% NOTE: the paragraph names can be removed later on
\section{Introduction}
With the imminent advent of consumer-level virtual and augmented reality technology, the ability to interact with the digital world with the most natural mean of interaction, our hands, becomes of paramount importance. Most importantly, the degree of immersion in the virtual world is directly correlated to whether the user perceives his own realistic hands~\todo{\cite{immersion}}. Over the past two decades the research community has explored a number of techniques to tackle this problem, from expensive and unwieldy marker-based mocap~\todo{\cite{mocapsurvey}} to instrumented gloves~\cite{dipietro2008survey} as well as imaging systems~\cite{erol2007survey}. Multi-camera imaging systems can recover the hand pose and hand-objects interactions with high accuracy~\cite{ballan2013salient}, but the only system capable to approach interactive applications is the ~10 fps system of~\cite{sridhar2013multicam}. Conversely, in this paper we focus on hand motion tracking with a \emph{single} RGBD sensor (e.g. Intel RealSense or Microsoft Kinect), as we foresee this is what it will be readily available in a typical AR/VR consumer experience.

\paragraph{Tracker: discriminative v.s. generative.}
Modern systems for real-time tracking from RGBD data  \cite{sridhar2015fast,sharp2015accurate} rely on a combination of \emph{discriminative} approaches like \cite{oberweger2015feedback}, and \emph{generative} approaches such as \cite{tagliasacchi2015robust}. The per-frame re-initialization of discriminative methods prevents error propagation by offering a continous recovery from tracking failure. As these discriminative models are learnt from data, they typically only estimate a coarse pose. Therefore, generative models are used to refine the estimate by aligning a geometric template of the user hand to the measured point cloud \AT{temporal?}. It is not surprising that the quality of the template directly affects the quality of pose refinement; see \Figure{teaser}. 
\TODO{mention no need to re-init $\gg$FPS?}.
% 
Therefore, the main goal of this paper is to explore novel tracking templates that strike an optimal \emph{balance} between accuracy and performance, that is, a model that is able to more accurately capture the user's geometry, while retaining the ability to answer registration queries in close form with very high efficiency. In \Figure{coarsemodel} and \todo{Video [00:00]} we illustrate the importance of employing a tracking template that strikes this delicate balance.

% \paragraph{Model calibration.}
% As generative models perform tracking by fitting a geometric model to what is measured by the sensor, our template should be able to represent well the observed data. However, even in modern trackers, the discrepancy between the optimal model pose given the data and the true hand pose can be significant. For example, the sources of discrepancy in \Figure{coarsemodel} include the incorrect length of fingers and the lack of the \todo{[pinky-ring]} degree of freedom. \todo{In the literature, the creation of user-specific models for tracking is referred to as \emph{calibration}.}

\begin{figure}[b!]
\centering
\begin{overpic} 
[width=\linewidth]
% [width=\linewidth,grid,tics=10]
{fig/coarsemodel/calibration}
% \put(75,10){{\Large TODO}}
\end{overpic}
\vspace{-.25in}
\caption{
% 
% 
(left) Tracking when the model from \protect\cite{tagliasacchi2015robust} is used without proper coarse scale calibration. (middle) A roughly manually calibrated model can help increasing the fitting fidelity, but tuning becomes increasingly difficult with more degrees of freedom. (right) The proposed automatically calibrated model. 
% 
% 
}
\label{fig:coarsemodel}
\end{figure}
\paragraph{Surface v.s. volumetric templates.}
In modern digital production representing objects by a piecewise linear meshing of their surface (i.e. triangular or quad meshes) is the de-facto standard. However, unlike volumetric models~\cite{bloomenthal1997book}, surface representations cannot efficiently answer queries such as the distance from the point to the object's boundary, or whether a point lies inside/outside the model~\cite[Ch.1]{botsch2010book}. In tracking applications these queries play a fundamental role, as the optimization attempts to find configurations where the average \emph{distance} from model to data is minimized. Similarly, a tracker should prevent the model from assuming implausible configurations, for example by preventing self-intersections as measured by inside/outside predicates. For all these reasons, volumetric models appear optimal for registration applications; indeed, recent compelling results in joint rigid registration and reconstruction~\cite{newcombe2011kinfu} as well as its recent non-rigid variant~\cite{newcombe2015dynfusion} leverage volumetric models. One important observations is that such techniques assume the frame-rate is high compared to user motion velocity, a condition that is surely not realizable in our setting. \AT{something about the fact that deforming a surface is easier than deforming a volume?} To tackle this challenge, in this paper we propose to employ a \emph{hybrid} model for tracking combining the advantages of surface and volumetric representations.

\begin{figure}[t!]
\begin{overpic} 
[width=\linewidth]
% [width=\linewidth,grid,tics=5]
{fig/convsurf/item.pdf}
\put(25,49){\small{$\mathcal{M}$}}
\put(74,49){\small{$\mathcal{M}$}}
\put(9,46){\small{$m_1$}}
\put(40,44.5){\small{$m_2$}}
\end{overpic}
\vspace{-.3in} 
\caption{
% 
% Thanks: ryoichi_sig13
The mesh {\small$\mathcal{M}$} explicitly identifies sphere positions and controls the union convolution operator to generate an implicit function. The zero-crossing of this implicit function describes the convex-hull of our spheres. \TODO{bug? bottom-left I cannot set the circle stroke size to zero, it becomes a dashed line}
% 
% 
}
\label{fig:convsurf}
\end{figure}


\paragraph{Hybrid tracking model.}
The model we propose in this paper is a variant of a convolution surface~\cite{bloomenthal1991convolution}, and its fundamental building block is illustrated in \Figure{convsurf}. Such a construct is nothing but the zero iso-surface of the scalar function:
\begin{equation}
\phi(\mathbf{x}) = \min \int_{\mathbf{c} \in \skeleton} \mathcal{B}_{\mathbf{c}, r(\mathbf{c})}(\mathbf{x}) \: d\mathbf{c},
\label{eq:convsurf}
\end{equation}
where $\skeleton$ is a skeletal control mesh (a segment or a triangle with respect to the simple example in \Figure{convsurf}), and $\mathcal{B}$ is the implicit function of a sphere parameterized by its center $\mathbf{c}$ and radii $r$:
\begin{equation}
\mathcal{B}_{\mathbf{c}, r(\mathbf{c})}(\mathbf{x}) = \|\mathbf{x}-\mathbf{c}\|^2 - r(\mathbf{c})^2.
\end{equation}
The spheres centers $\mathbf{c}$ span the skeleton $\skeleton$, while the radii is a function of the position $\mathbf{c}$ within an element, and more specifically, it is linearly interpolated from values specified on the skeletal mesh vertices $r_*=r(c_*)$. This is indeed a \emph{hybrid} model, as \Eq{convsurf} defines an implicit surface $\mathcal{S} = \{\mathbf{x} \in \mathbb{R}^n | \phi(\mathbf{x})=0 \}$, while the underlying skeleton $\mathcal{M}$ is an explicit representation (a piecewise-parameterized \todo{surface}). We generalize this basic construct to devise a model suitable to represent a human hand; see \Figure{topology}.
\todo{While the integral in \Eq{convsurf} might be difficult to evaluate, distances to $\mathcal{S}$ can conveniently be computed by querying distances to the piecewise linear structures composing $\mathcal{M}$; see \Figure{convsurf}.}

\begin{figure}[h]
\centering
\begin{overpic} 
[width=\linewidth]
% [width=\linewidth,grid,tics=10]
{fig/topology/item.pdf}
%{fig/topology/topology}
% \put(10,10){\todo{fig:topology}}
\end{overpic}
\caption{(left) the underlying skeleton $\skeleton$ that parametrizes the convolution surfaces and a representation of the radii defined at each vertex, solid parts are shown in dark green, articulated parts in purple and elastic parts in light green.
 (right) the surface $\surface$ can be ray-traced in the fragment shader, triangulated or rendered as a set of capsule models \Figure{convsurf}-(right). \TODO{anastasia: only have 2x rendering -- the two in the left column to be precise. Also it would be a good idea to change the colors of segments v.s. spheres, no?} \TODO{color-code rigid parts of topology, and mention it in the caption} }
\label{fig:topology}
\end{figure}
\paragraph{Tracking and calibration with convolution models.}
Our novel tracking model has two significant advantages. (1) First of all, distance queries to $\surface$ can be executed by measuring the distance to the skeletal structure $\skeleton$. The number of elements in $\skeleton$ is significantly smaller (\todo{25} in our model) than the number of polygons in a typical triangular mesh representation of $\surface$~\cite{thiery2013sphere}. Therefore, not only our distance queries can be executed with great efficiently using a brute force approach, but this leads to an algorithm that is trivially parallelizable and that executes with a fixed frame-rate. (2)~Our hand model parameterization is also compact, as we can generate a family of models by simply adjusting \emph{positions} and \emph{radii} of the control skeleton vertices $c_* \in \skeleton$. A direct approach to building a calibrated model is to let the artist manually adjust these quantities; this process is known as  \emph{ZSphere~{\textcopyright}} modeling in the popular \emph{ZBrush~\textcopyright} 3D modeling software. \TODO{Something about why this makes the model efficient.} \TODO{Something about many recent SphereMeshes and ImplicitSkinning papers at SIGGRAPH?}

\begin{figure}[t!]
\centering
\begin{overpic} 
[width=\linewidth]
% [width=\linewidth,grid,tics=10]
{fig/handmodels/item.pdf}
\put(03,2){\small (a)}
\put(24,2){\small (b)}
\put(51,2){\small (c)}
\put(81,2){\small (d)}
\end{overpic}
\caption{
% 
%
Tracking templates used in recent generative approaches methods: (a)~\protect\cite{qian2014realtime}, 
(b)~\protect\cite{oikonomidis2014evolutionary},
(c)~\protect\cite{melax2013dynamics}, and
(d)~\protect\cite{sharp2015accurate}.
% 
% 
}
\label{fig:handmodels}
\end{figure}


\paragraph{Contributions.}
% Technical Contribution: What's the key technical idea to solve the problem? Why is it beautiful?
% Applications / Future Work: What will your solution enable? How does it project into the future? How will it inspire future work?
In our research we we identify a number of contributions:
(1)~We demonstrate that convolutional models can be effectively employed for tracking of geometry in motion. 
(2)~Amongst many possible variants, we identify an effective convolution surface  topology for hand-tracking. \AT{won't they us to compare against other topologies?}
(3)~We demonstrate that the increased model accuracy does not aberrate tracking performance, leading to a 60 FPS tracking algorithm.
% (3) While any geometric model could be employed in tracking applications, it is imperative to determine whether a model is suitable for real-time tracking applications; we demonstrate this is the case by integrating our models in an open-source tracking system~\cite{tagliasacchi2015robust} and retaining 60 FPS tracking performance. 
(4)~We introduce an optimization technique for convolution models that adapts a given template to a given user, and demonstrate how this results in substantial improvements in tracking precision.
(5)~Rather than relying on a artist-built IK skeleton, we define an optimization problem to automatically identify the kinematic decomposition of motion from data.
%
\hspace{0in}
% 
%--- Areas of future work
Our findings pave the way to a number of interesting venues for future works, including (1)~the extension of our modeling framework to the capture of generic articulated geometry in motion, (2)~the consolidation of optimization for modeling and tracking stages, 
(3)~the generation of parameterizations to enable texturing and level-of-detail representations, 
(4)~the possibility to 
% 



% \item Proposed a single optimization framework for modeling and tracking
% \item Skinning. Display the texture on the model. For that a  parametrization is required. Potentially create ``bumps map'' to apply the errors from model fitting stage
% \item Automatically identifies the degrees of freedom and axes of rotation from data
