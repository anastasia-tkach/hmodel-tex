% Problem Statement: What's the problem you want to solve?
% 
% Motivation: Why is this an interesting problem? Who cares about it? Why now? Why is it appropriate for the conference audience?
% 
% Research Gap, Novelty: Why is new research required? Why can the problem not be solved with existing methods? How does the proposed solution differ from and/or improve upon existing work?
% 
% Technical Contribution: What's the key technical idea to solve the problem? Why is it beautiful?
% 
% Applications / Future Work: What will your solution enable? How does it project into the future? How will it inspire future work?
\section{Introduction}

%--- Why hand-tracking is fundamental (something abour AR/VR?)
% Hand tracking is a process of accurately reconstructing shape and articulation of human hands. It is a crucial component of natural human-computer interfaces and animation of humanoid avatars.
\TODO{In our world, our hands are our main mean of interaction, and with the advent of augmented and virtual reality experiences, there is a compelling need ...}
%--- Camera-based tracking & Physics
Accurate real-time hand tracking is therefore a fundamental technical challenge \todo{In AR and VR applications a 3D hand model can properly interact with 3D objects, establish a realistic contact and disappear behind them. Given that, the degree of immersion into virtual reality depends on whether a user sees own realistic hands (find a study mentioned by Leap Motion).}

\TODO{the paragraph names can be removed later on}

%--- Combination of model+appearance
\paragraph{Tracker: discriminative v.s. generative.}
Modern systems for real-time tracking \cite{sridhar2015fast,sharp2015accurate},  rely on a combination of \emph{discriminative} approaches like \cite{oberweger2015feedback}, and \emph{generative} approaches such as \cite{tagliasacchi2015robust}. The per-frame re-initialization of discriminative methods prevents error propagation by offering a continous recovery from tracking failure. As these discriminative models are learnt from data, they typically only provide a coarse and jittery pose. Therefore, generative models are used to refine the estimate by aligning a geometric template of the user hand to the measured point cloud \AT{temporal?}. It is not surprising that the quality of the template directly affects the quality of pose refinement; see \Figure{teaser}. 
\TODO{mention that no need to re-init at high FPS?}.
% 
Therefore, the main goal of this paper is to explore novel tracking templates that strike an optimal \emph{balance} between accuracy and performance, that is, a model that is able to accurately capture the user's geometry, while ...

 

 between the ability to accurately model the hand geometry, 

while having little impact on the computational complexity of a   

\paragraph{Surface v.s. volumetric models.}
\todo{In digital production triangular meshes are the de-facto standard geometric representation, they are difficult to employ in real-time system.}


% \TODO{why is it necessary, plus previous works}
\paragraph{Model calibration.}
\begin{figure}[b!]
\centering
\begin{overpic} 
[width=\linewidth]
% [width=\linewidth,grid,tics=10]
{fig/coarsemodel/calibration}
% \put(75,10){{\Large TODO}}
\end{overpic}
\vspace{-.25in}
\caption{
% 
% 
(left) Tracking when the model from \protect\cite{tagliasacchi2015robust} is used without proper coarse scale calibration. (middle) A roughly manually calibrated model can help increasing the fitting fidelity, but tuning becomes increasingly difficult with more degrees of freedom. (right) The proposed automatically calibrated model. 
% 
% 
}
\label{fig:coarsemodel}
\end{figure}
As generative models perform tracking by fitting a geometric model to what is measured by the sensor, our template should be able to represent well the observed data. However, even in modern trackers, the discrepancy between the optimal model pose given the data and the true hand pose can be significant. For example, the sources of discrepancy in \Figure{coarsemodel} include the incorrect length of fingers and the lack of the \todo{[pinky-ring]} degree of freedom. \todo{In the literature, the creation of user-specific models for tracking is referred to as \emph{calibration}.}

\paragraph{LBS Looks Crappy}
\todo{The hand skinning quality is obviously important for digital avatars applications. The simple skinning approaches like linear blend skinning may generate implausible results.}

\begin{figure}[t!]
\centering
\begin{overpic} 
[width=\linewidth]
% [width=\linewidth,grid,tics=10]
{fig/handmodels/item.pdf}
\put(03,2){\small (a)}
\put(24,2){\small (b)}
\put(51,2){\small (c)}
\put(81,2){\small (d)}
\end{overpic}
\caption{
% 
%
Tracking templates used in recent generative approaches methods: (a)~\protect\cite{qian2014realtime}, 
(b)~\protect\cite{oikonomidis2014evolutionary},
(c)~\protect\cite{melax2013dynamics}, and
(d)~\protect\cite{sharp2015accurate}.
% 
% 
}
\label{fig:handmodels}
\end{figure}



\TODO{--- STOPPED EDITING HERE ---}

\begin{figure*}[t]
\centering
\begin{overpic} 
% [width=\linewidth]
[width=\linewidth,grid,tics=10]
{fig/twocol/item.pdf}
\put(10,10){\todo{\Large Overlay Text}}
\end{overpic}
% \caption{{\color{red}[onecol]}}
\label{fig:twocol}
\end{figure*}

\subsection{Alternative hand model representations}
We suggest to use convolution surfaces representation of the hand model. Convolution surface is an implicit surface which is described by a control skeleton. The skeleton may consist of points, edges or polygons \cite{bloomenthal1991convolution}. In each vertex of the skeleton we define a radius. The radius in intermediate points is a linear combination of the radii at the neighboring vertices. Given the topology of the underlying skeleton, the model can be represented with convolution surface up to high precision \textcolor{mygray}{(find some theoretical estimates).} Next we present the arguments why convolution surfaces representation is suitable for all the stages of the pipeline.

\begin{table}[!ht] 
	\centering
	\begin{tabular}{|p{2.5cm}|p{2.5cm}|p{2.5cm}|}
	\hline
 	& Hand pose  & Hand shape  \\
	\hline
	Triangular mesh with embedded skeleton, \cite{taylor2014user} & Vertices and bones positions & Vertices and bones positions	 \\
	\hline
	Cylinder model, \cite{tagliasacchi2015robust} & Cylinders size and transformations & Cylinders transformations	 \\
	\hline
	Convolution surfaces model & Positions and radii of control points & Positions of control points \\
	\hline
	\end{tabular}
	\vspace{1em}
	\caption{Comparison of different hand model representations}
	\label{table:representation_dependent_components}
\end{table}

\subsection{Convolution surfaces for model fitting}
The spheres and mixed cylinders/spheres hand model representations (Figure \ref{fig:hand_model_representations} a, b) are ubiquitous in hand tracking, because they are well suited for tracking tack per se (see next) and can be quickly to created manually. If a small number of  building blocks is used, the precision of the model is low, especially in the palm region. A higher precision can be obtained by increasing the number of primitives, which defeats the purpose of model simplicity. Convolution surfaces representation gives higher precision for the same number of building blocks. \textcolor{mygray}{Add experimental or theoretical support for convolution surfaces.}
The triangles mesh representation can approximate the hand to a high precision. However, it is not trivial to specify the model parts that are rigid and should be kept the same shape between the poses. This results in overfitting to local skin deformations. 

\subsection{Convolution surfaces for hand tracking}
For model based tracking the main operation is to find the closest point on the model for a given data point. This operation is can be done in closed for each rigid segment with spheres/cylinder and convolution surfaces model representation. 
For a triangular mesh this operation has complexity linear in number of triangles. Thus, it is necessary to simplifying assumptions and/or more complex optimization to allow the hand tracking system to run in real time. (Look how different systems deal with this problem). Moreover, the triangular mesh has (much) more degrees of freedom than the underlying problem. Without additional regularization, rigid parts of the hand model can deform to fit the data and the individual vertices can shift to fit the sensor noise.

\subsection{Convolution surfaces for hand skinning}
The Linear Blend Skinning approach used to pose the triangular mesh model in previous works (\cite{sharp2015accurate}, \cite{schroder2013analysis} ) creates artifacts, the fingers looks like made from rubber. The spheres/cylinders model is not suitable for realistic animation, therefore a re-targeting step to a template mesh is required. Retargeting does not only demand additional effort, but also brings additional imprecision. The state of the art approaches in hand skinning are implicit surfaces-based (\cite{vaillant2013implicit},  \cite{vaillant2014robust} ).  A convolution surfaces model serves a ready to use input for such an approach.

\subsection{Contributions}

\subsubsection*{1. Suggested a hand model representation well suited for user-specific adaption, tracking and animation}

TODO: 

\begin{itemize}
\item Modeling. Implement initialization for the model fitting by running Htrack and putting the hand in a similar pose. 
\item Skinning. Display the texture on the model. For that a parametrization is required. Potentially create ``bumps map'' to apply the errors from model fitting stage. 
\end{itemize}

\subsubsection*{1a. Proposed a single optimization framework for modeling and tracking}

TODO:

\begin{itemize}
\item Run a couple of iteration of ARAP for palm after IK optimization just with data term.
\end{itemize}

\subsubsection*{1b. Developed a tracking model that represents all the degrees of freedom of human hand without creating inexistent degrees. (Not guaranteed by the other approaches, look how model adaptation paper finds the degrees of freedom)}

\subsubsection*{Improved accuracy of state-of-art hand tracking system}

TODO: 

\begin{itemize}
\item Add outline term, time interpolation of initial joint angles, downsampling of the sensor data, fingertips term.
\end{itemize}