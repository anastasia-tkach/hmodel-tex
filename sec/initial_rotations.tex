\subsection{Initial Rotations}
The overwhelming majority of hand  tracking systems uses joint angles as an optimization variable. Given the vector of joint angles $\theta$, one also needs to know the initial rotations of the fingers with respect to the palm to be able to pose the model. (This is especially relevant for user-specific model adaptation). The precise tracking is impossible without having the correct initial rotations, because the joints at the base of fingers cannot rotate around their axis, thus initial rotations constrain the possible poses that a hand model can assume. The initial rotations of the fingers can be guessed approximately because the are close to identity, but with the thumb it is much more difficult. \Anastasia{To prove this, I could alter the rotations in our system and show that tracking gets worse.} 
To compute initial rotations we formulate an optimization problem for all the input hand poses. Denote the set of model centers in pose $i$ as $C_i = \{c_1^i, ... c_{|C_i| }^i\}$. As a preprocessing step we compute a rigid rotation between the rigid palm centers (the centers that are not articulated or elastic) and rotate all the poses to the same position, such that base centers of their fingers match together.

The initial rotation is computed separately for each finger, so let us just discuss one finger. We minimize the norm of the difference between the centers of the finger at the current $i$ pose $\{c_1^i, c_2^i, c_3^i \}$ and the positions of the centers posed using the same initial rotations across all poses $\{\hat{c}_1^i, \hat{c}_2^i, \hat{c}_3^i \}$. Here $c_1^i$ stands for the base of the second phalange, $c_2^i$ stands for the base of the third phalange, and $c_3^i$ the end of the third phalange.

Denote the initial rotations of the first, second and third phalanges as $I_1$, $I_2$ and $I_3$. We parametrize initial rotations of a finger by 5 values $\alpha = \{\alpha_1, ..., \alpha_5\}$, such that $I_1 = R_z(\alpha_3) R_y(\alpha_2) R_x(\alpha_1).$ \Anastasia{it is in reversed order to much the way the rotations are applied in Htrack, probably this should not  go to the paper} $I_2 = R_z(\alpha_4)$ and $I_3 = R_z(\alpha_5)$, where $R_x$, $R_y$ and $R_z$ are rotations around axis $X$ (side), $Y$ (up) and $Z$(front). We choose not to allow initial rotations around $X$ and $Y$ for the second and third segment partially to decrease the smallest number of required hand poses, partially because it seems that this rotations do not exist in real hand.

Denote the rotations given by the values of the joint angles for the pose $i$ for the phalanges as $J_1^i$, $J_2^i$ and $J_3^i$.
\begin{equation*}
J_1^i = R_x(\theta_2^i) R_z(\theta_1^i),
J_2^i = R_x(\theta_3^i),
J_3^i =  R_x(\theta_4^i),
\end{equation*}
where $\theta_1^i$ and $\theta_2^i$ are abduction and flexion of the first phalange and $\theta_3^i$ and $\theta_4^i$ are flexions of the second and third phalanges at the pose $i$. 
The positions of the centers posed with the same initial transformations are computed as following:
\begin{align*}
& \hat{c}_1^i =  t_1 + I_1   J_1^i    l_1  u, \\
& \hat{c}_2^i = t_1 + I_1    J_1^i   (t_2 + I_2  J_2^i  l_2  u), \\
&\hat{c}_3^i =  t_1 + I_1    J_1^i   (t2 +  I_2   J_2^i  (t_3 + I_3  J_3^i  l_3 u)),
\end{align*}
where $t_1$, $t_2$, $t_3$ and $l_1$, $l_2$, $l_3$ are the translations and length of the bases of the phalanges 1, 2 and 3. They are computed from the input locations of the centers are constants; $u$ is a vector along the axis $Y$ (up).
