\section{Related Works}

% MORE STUFF
% http://files.is.tue.mpg.de/dtzionas/In-Hand-Scanning/

\TODO{andrea}
\todo{Lorem ipsum dolor sit amet, consectetur adipisicing elit, sed do eiusmod tempor incididunt ut labore et dolore magna aliqua. Ut enim ad minim veniam, quis nostrud exercitation ullamco laboris nisi ut aliquip ex ea commodo consequat. Duis aute irure dolor in reprehenderit in voluptate velit esse cillum dolore eu fugiat nulla pariatur. Excepteur sint occaecat cupidatat non proident, sunt in culpa qui officia deserunt mollit anim id est laborum.}
\newpage
\todo{Lorem ipsum dolor sit amet, consectetur adipisicing elit, sed do eiusmod tempor incididunt ut labore et dolore magna aliqua. Ut enim ad minim veniam, quis nostrud exercitation ullamco laboris nisi ut aliquip ex ea commodo consequat. Duis aute irure dolor in reprehenderit in voluptate velit esse cillum dolore eu fugiat nulla pariatur. Excepteur sint occaecat cupidatat non proident, sunt in culpa qui officia deserunt mollit anim id est laborum.}
\endinput

\TODO{\cite{bogo2015detailed}}

%--- ON SPHERE MODELS
\TODO{A higher precision can be obtained by increasing the number of primitives, which defeats the purpose of model simplicity. Convolution surfaces representation gives higher precision for the same number of building blocks. \textcolor{mygray}{Add experimental or theoretical support for convolution surfaces.}}

%--- LINEAR BLEND SKINNING
\TODO{The Linear Blend Skinning approach used to pose the triangular mesh model in previous works (\cite{sharp2015accurate}, \cite{schroder2013analysis} ) creates artifacts, the fingers looks like made from rubber. The spheres/cylinders model is not suitable for realistic animation, therefore a re-targeting step to a template mesh is required. Retargeting does not only demand additional effort, but also brings additional imprecision. }

%--- IMPLICIT SKINNING
\TODO{The state of the art approaches in hand skinning are implicit surfaces-based (\cite{vaillant2013implicit},  \cite{vaillant2014robust} ).  A convolution surfaces model serves a ready to use input for such an approach.}

%--- CLOSEST POINT COMPUTATIONS
\TODO{For model based tracking the main operation is to find the closest point on the model for a given data point. This operation is can be done in closed for each rigid segment with spheres/cylinder and convolution surfaces model representation. For a triangular mesh this operation has complexity linear in number of triangles. Thus, it is necessary to simplifying assumptions and/or more complex optimization to allow the hand tracking system to run in real time. (Look how different systems deal with this problem). Moreover, the triangular mesh has (much) more degrees of freedom than the underlying problem. Without additional regularization, rigid parts of the hand model can deform to fit the data and the individual vertices can shift to fit the sensor noise.}

%--- JUST A LIST OF PAPERS?
\todo{A number of hand tracking algorithms has been recently proposed  Keskin et. al. \cite{keskin2012hand}, Melax et. al. \cite{melax2013dynamics}, Tang et. al. \cite{tang2013real}, Oikonomidis et. al. \cite{oikonomidis2014evolutionary}, Schroder et. al. \cite{schroder2014real},
Tompson et. al. \cite{tompson2014real}, Qian et. al. \cite{qian2014realtime},  Tagliasacchi et. al. \cite{tagliasacchi2015robust}, Sridhar et. al. \cite{sridhar2015fast}, Sun et. al. \cite{sun2015cascaded} and Sharp et. al. \cite{sharp2015accurate}.}

\begin{itemize}

\item Albrecht et al. \cite{albrecht2003construction} developed an approach for creating an anatomically realistic hand model that includes bones and muscles structure. Their approach requires several prerequisites including plaster cast of a human hand and laser scanner for manually creating a physically realistic hand template. Given user-defined correspondences between 3D feature points and the hand image, a specific hand model is created by deforming a generic hand model. 

\item Rhee et. al. \cite{rhee2006human} use a single image of a hand at rest pose to infer joint hand joint locations from skin creases. Given the skeleton obtained at the previous step and the hand contour from the image, they deform a template hand mesh to fit this data. 

\item Straka et al. \cite{straka2012simultaneous} also fit the template mesh with attached skeleton to 3D data. The model is deformed to explain the data while keeping the vertices attached to their corresponding bones. It is on clear whether the approach whether be able to handle a hand motion sequence, since the results are demonstrated on a full body model.

\item Taylor et. al. \cite{taylor2014user} generate a user-specific hand model from an RGBD video sequence. The model is represented as a triangular mesh with an embedded skeleton. In each frame the hand pose is initialized using an appearance-based tracking algorithm. The hand model parameters are found by solving a single optimization problem formulated for the entire video sequence which also finds hand pose in each frame. 

\item Khamis et. al.  \cite{khamis12learning} fit a hand model for a specific user by finding its shape coordinates in the basis of mesh matrices and bones locations. As in the approach by Taylor et. al, they optimize simultaneously for pose and shape parameters in all the frames of an RGBD sequence across all the subjects. Requires large number of subjects as a regularization for excessive degrees of freedom. 

The results generated by the approaches listed above could be used as an input for our system to create a hand model representation adapted for efficient tracking and animation.


\end{itemize}