\section{Related Works}

\todo{A number of hand tracking algorithms has been recently proposed  Keskin et. al. \cite{keskin2012hand}, Melax et. al. \cite{melax2013dynamics}, Tang et. al. \cite{tang2013real}, Oikonomidis et. al. \cite{oikonomidis2014evolutionary}, Schroder et. al. \cite{schroder2014real},
Tompson et. al. \cite{tompson2014real}, Qian et. al. \cite{qian2014realtime},  Tagliasacchi et. al. \cite{tagliasacchi2015robust}, Sridhar et. al. \cite{sridhar2015fast}, Sun et. al. \cite{sun2015cascaded} and Sharp et. al. \cite{sharp2015accurate}.}

\begin{itemize}

\item Albrecht et al. \cite{albrecht2003construction} developed an approach for creating an anatomically realistic hand model that includes bones and muscles structure. Their approach requires several prerequisites including plaster cast of a human hand and laser scanner for manually creating a physically realistic hand template. Given user-defined correspondences between 3D feature points and the hand image, a specific hand model is created by deforming a generic hand model. 

\item Rhee et. al. \cite{rhee2006human} use a single image of a hand at rest pose to infer joint hand joint locations from skin creases. Given the skeleton obtained at the previous step and the hand contour from the image, they deform a template hand mesh to fit this data. 

\item Straka et al. \cite{straka2012simultaneous} also fit the template mesh with attached skeleton to 3D data. The model is deformed to explain the data while keeping the vertices attached to their corresponding bones. It is on clear whether the approach whether be able to handle a hand motion sequence, since the results are demonstrated on a full body model.

\item Taylor et. al. \cite{taylor2014user} generate a user-specific hand model from an RGBD video sequence. The model is represented as a triangular mesh with an embedded skeleton. In each frame the hand pose is initialized using an appearance-based tracking algorithm. The hand model parameters are found by solving a single optimization problem formulated for the entire video sequence which also finds hand pose in each frame. 

\item Khamis et. al.  \cite{khamis12learning} fit a hand model for a specific user by finding its shape coordinates in the basis of mesh matrices and bones locations. As in the approach by Taylor et. al, they optimize simultaneously for pose and shape parameters in all the frames of an RGBD sequence across all the subjects. Requires large number of subjects as a regularization for excessive degrees of freedom. 

The results generated by the approaches listed above could be used as an input for our system to create a hand model representation adapted for efficient tracking and animation.


\end{itemize}