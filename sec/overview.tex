\section{Overview}
\label{sec:overview}

\subsection{Review of HTrack}
\TODO{more details, with energies and all the computational elements we need for our approach (e.g. correspondence search, silhouette, etc.)}

%The main  we investigate in this paper is whether convolutional surfaces are a viable alternative for realtime hand tracking with
%single-view RBGD sensors. 

%Our hypothesis is that the increased geometry accuracy compared with cylinder models allows for a higher tracking precision without impairing the computational performance of the tracking algorithm.

%Essential for accurate tracking is sufficiently precise modeling of the hand geometry of the user. This requires a geometric representation that on the 

%The model is flexible in that it can easily be adapted to a specific user.

% \paragraph{Notation} \\
%% Sensor point cloud: $\point \in \PointsSensor$
%
%\paragraph{Calibration} \\
%\TODO{something...?}
%
%\paragraph{Tracking} \\
%Mention the ICP algorithm... \\
%We employ a Gauss-Newton approach... \\
%\TODO{rough description of htrack}