% !TEX root = ../hmodel.tex
\section{Results}
\label{sec:results}
We evaluate our technique on a variety of sequences across a number of users, 
% \Anastasia{across just one user}
% \AT{sentence above talks about calibration+tracking together. It's carefully written to make the unattentive reader believe it has several users}
and performed qualitative as well as quantitative  
% \Anastasia{only qualitative for all except for Htrack}
% \AT{I know but overall we do both qualitative and quantitative, no need to say what is what at this stage}
comparisons of our method to the state-of-the-art~\cite{qian2014realtime,sridhar2015fast,tagliasacchi2015robust,sharp2015accurate,taylor2016concerto}. We also propose new algorithm-agnostic metrics tailored to high-precision tracking evaluation, and release the new \handy{} dataset associated with these metrics.

\paragraph{Calibration}
The calibration of the convolution model to a collection of 3D data frames is illustrated in \Figure{calibration}; note the same model is rigidly articulated to fit to multiple poses. While for this user we build a model from multi-view stereo data (omni-directional, complete), it is important to notice that the use of multiple frames in different poses is a necessity. Only in this situation can the centers locations $\posedcenters$ be automatically adjusted to create an articulated model that consensually fits the whole dataset. We refer the reader to \Video{00:00} for a visualization of our iterative calibration procedure. The calibration from RGBD datasets can also be seen at \Video{00:00}, and the resulting models are illustrated in \Figure{topology}. 

The importance of adjusting kinematic chain transformations is  shown in \Figure{posing}. With incorrect  transformations, joint limits and the articulation restrictions of the kinematic chain can prevent the model from being posed correctly; see a dramatization in \Video{00:00}. In our experiments we discovered it was crucial to identify the \emph{typical} kinematic chain structure using the dataset in \Figure{calibration}; user-specific calibration optimization used these transformations as initialization.
% relatively fine-scale difference exist across users per-user adaptation.
% \MP{The last sentence is not clear to me.  Do you mean that one MVS model was necessary, but that the other users can be adapted with RGBD data?}

\paragraph{Comparison metrics}
Taylor and colleagues \shortcite{taylor2016concerto} have recently reported how state-of-the-art hand tracking algorithms have reached human precision in determining the location of key features (e.g. fingertips and wrist position). Therefore, publicly available datasets like \cite{tompson2014real} and \cite{sridhar2013multicam}, often relying on human labeling of data, have become less suitable for quantitatively evaluating the quality of high-precision tracking. 
% 
Therefore, we propose two easy-to-compute metrics to evaluate the quality of generative tracking algorithms. A core element that makes these metrics appealing is that, much like key feature positions, they are completely \emph{algorithm agnostic}. This is essential, as it will enable the research community to validate and compare results through quantitative analysis. 
% 
We achieve this goal by expressing these metrics exclusively as a function of the acquired depth image $\depth_n$, and of the depth image $\depthrend_{n}$ of the rendered model. Below we drop the subscript $n$ for notational brevity and only consider points  $\point \in \depth$ within the RoI.  Sharp et al.~\shortcite{sharp2015accurate} proposed the \emph{golden energy}, which is similar in spirit, but does not directly encode an approximation of the previously discussed (monocular) Hausdorff distance like our two metrics do. The first is the data-to-model asymmetric approximate Hausdorff metric: 
% 
\begin{equation}
\metricone = |\depth|^{-1} \sum_{\point \in \depth} \| \point - \proj_\depthrend(\point) \|_2^1
\label{eq:metric1}
\end{equation}
% 
Differently from before, $\proj_\depthrend$ computes the (kd-tree accelerated) closest point correspondence to the rendered model \emph{point cloud}, rather than to the model itself. The second asymmetric metric is evaluated in the 2D projective space:
% 
\begin{eqnarray}
\metrictwo = |\depthrend \setminus \partial\depth|^{-1} 
\sum_{\pixel \in \depthrend \setminus \partial\depth} \| \pixel - \proj_{\partial\depth}(\pixel) \|^1_2
% &=& |\depthrend \setminus \partial\depth|^{-1}
% \sum_{\pixel \in \depthrend \setminus \partial\depth} \text{DT}(\pixel)
\label{eq:metric2}
\end{eqnarray}
% 
Each summation term above can be evaluated efficiently by pre-computing the 2D Euclidean distance transform of the RoI's (image-space) silhouette $\partial \depth$, where the transform evaluates to zero for a pixel inside the silhouette.

\begin{figure}[t!]
\centering
\begin{overpic}
[width=\linewidth]
% [width=\linewidth,grid,tics=10]
{\currfiledir/item.pdf}
\put(1,30){{\small$\%$}}
\put(1,64){{\small$\%$}}
\put(93,37.5){{\small $E_{3D}$}}
\put(93,04.5){{\small $E_{2D}$}}
\end{overpic}
\caption{
% 
%
\todo{Each plot visualizes on the $y$ axis the portion of frames with a mean error metric below the value reported on the $x$ axis. We employ the \handyseq{teaser} sequence for this purpose. Curves closer to the top-left quadrant indicate better performance.}
% 
% Our method is quantitatively compared to the one of~\protect\cite{tagliasacchi2015robust} on the \handyseq{teaser} sequence.
% 
% 
}
\label{fig:comp2}
\end{figure}
\paragraph{Handy dataset}
We create the new \handy{} tracking dataset for the evaluation of high-precision generative tracking algorithms. Our dataset contains \todo{$\approx$10m} of recorded sequences from an \realsense{} SR300 sensor. The dataset is designed to cover the entire range of motions that has been surveyed in recent techniques. As detailed in \Figure{motiontypes}, we identified three main axes of complexity in the hand tracking literature, and devised the \textsc{teaser} dataset to thoroughly sample this space; see~\Video{00:00}. 
% 
% \AnastasiaComment{We should add, that there is at least one more axis of complexity, which is quality of sensor data. Sharp et. all have explored that axis a lot. They have sequences for far away (which our sensor does not allow) and sequence where the hand is only partially visible (not possible for us, because we use purely model base tracking). Anyway, system is not intended to be used in such scenarios, if a person wants to have a hand tracked, he or she should make sure that the sensor can see the hand. AT: I think it's quite ok, perhaps we can add it to the revised version (after reviews)?}
% 
Further, to enable qualitative comparisons to motions from state-of-the-art papers we also devised an additional set of sequences:
% 
\begin{description}[labelsep=0em,labelwidth=1.6in,labelindent=1cm,itemsep=-.6em]
    \item[\VideoExtra{00:00} -- \textsc{tayl1}] rigid and clenching
    \item[\VideoExtra{00:00} -- \textsc{srid1}] finger extension 
    \item[\VideoExtra{00:00} -- \textsc{srid2}] fingers contact
    \item[\VideoExtra{00:00} -- \textsc{srid3}] crossing fingers
    \item[\VideoExtra{00:00} -- \textsc{srid4}] pinching
    \item[\VideoExtra{00:00} -- \textsc{shar1}] fast and \todo{unlikely} 
    \item[\VideoExtra{00:00} -- \textsc{shar2}] fast rigid 
    \item[\VideoExtra{00:00} -- \textsc{shar3}] rotating fist
\end{description}
The sequences marked as \emph{tayl*}, \emph{srid*} and \emph{shar*} are respectively designed to emulate the motions in~\cite{taylor2016concerto}, \cite{sridhar2015fast} and~\cite{sharp2015accurate}. We do not devise sequences for~\cite{qian2014realtime} and \cite{tompson2014real} as the previous datasets already covered the motion space.
% 
% \AnastasiaComment{We do not use Qian et al and Tompson et al. because Taylor, Sridhar and Sharp already cover motion space more or less. Also, too many sequences could not fit on the plot.}

\paragraph{Self evaluation}
% \AnastasiaComment{In figure 12, the X axis numerical values are not visible for E3D and are visible for E2D. You decided to remove ``no data energy'' from the plots. It was at bottom limit for E3D and at top limit for E2D. I think, it illustrates nicely why silhouette energy makes E3D metric worse, since data energy also make E2D metrics MUCH worse.}
% 
In~\Figure{comp2}, we adopt the self-evaluation visualization proposed by \cite{taylor2016concerto}. We study the changes in algorithm performance as we disable the tracking energy terms in \Equation{htrack} on the \handyseq{teaser} sequence -- in all tests the \emph{d2m} term is never disabled as otherwise immediate loss of tracking occurs. Not surprisingly, we identify the \emph{limits} and \emph{pose} terms to be the ones dominating performance, as they directly encode the look and feel of a human hand. The 2D metric is clearly dominated by the \emph{m2d} fitting term, while this has little effect on the 3D metric. 
% 
% \AnastasiaComment{maybe say again that both E2D and E3D are essential to judge about the system performance, because they are two sides of the distance between two sets. AT: we talk about this already a number of times! a few lines above actually.}


\begin{figure}[t!]
\centering
\begin{overpic} 
[width=\linewidth]
% [width=\linewidth,grid,tics=10]
{\currfiledir/item.pdf}
% \put(0,-3){\todo{\currfiledir}}
\put(6,97){{\small $E_{3D}$ }}
\put(68.4,96.5){{\tiny \cite{tagliasacchi2015robust}}}
\put(73,94.5){{\tiny [Proposed Method]}}
\put(6,47.5){{\small $E_{2D}$ }}
\put(68.4,47.4){{\tiny \cite{tagliasacchi2015robust}}}
\put(73,45.4){{\tiny [Proposed Method]}}
\end{overpic}
\caption{
% 
%
\handyseq{teaser}
% 
% 
}
\label{fig:comp1}
\end{figure}
\paragraph{Quantitative comparison}
Our algorithm has been tested with the \realsense{} SR300 (QVGA@60Hz).  We have tailored the method of~\cite{tagliasacchi2015robust} to support this sensor to enable quantitative comparisons. In \Figure{comp1}, our two metrics are plotted per-frame as the two tracking algorithms are executed on the \handyseq{teaser} sequence; see \Video{00:00}. Aggregated performance comparisons are also reported in \Figure{barchart} for each sequence in the \handy{} dataset; see \VideoExtra{00:00}. These metrics reveal a consistent and significant increase in performance; this performance gain can be better appreciated in our accompanying videos. \Figure{calibeval} quantitatively illustrate the benefits of a calibrated template.

Given a sufficiently rich annotated data sample, it is generally possible to adapt a discriminative tracker to a different sensor from what it was originally designed for. However, for generative algorithms the task requires some parameter tweaking, something extremely challenging to achieve without direct access to each sensor variant. For these reasons, comparing to datasets developed on different sensors like \dexter{}~\cite{sridhar2013multicam} or \fingerpaint{} \cite{sharp2015accurate} is simply not feasible. Most importantly, these datasets were acquired at 30Hz, while our generative algorithm is specifically designed to execute at 60Hz. To enable a fair comparison, we would require the discriminative per-frame re-initializer employed by the authors, but no sources for these algorithms are available.

\paragraph{Qualitative comparison}
While we do not perform quantitative comparisons, we employ the \handy{} sequences to perform a qualitative comparison to~\cite{qian2014realtime,sridhar2015fast,sharp2015accurate,taylor2016concerto}. As it can be observed in~\VideoExtra{00:00}, our calibrated tracker is capable to replicate any of the motions benchmarked by state-of-the-art techniques with excellent accuracy.


% \paragraph{Golden Energy}
% Another metric we consider is the golden energy from Sharp and colleagues~\shortcite{sharp2015accurate} that was employed to drive their PSO real-time tracker:
% \begin{equation}
% E_{\text{Au}} = \sum_{ij} \rho(\depth(i,j) - \depthrend(i,j))
% \label{eq:golden}
% \end{equation}
% where $\rho(e)=\min(|e|,\tau)$ is a truncated $\ell_1$ distance to make the metric less sensitive to outliers ($\tau=100mm$). Pixels outside the region of interest are mapped to the far plane $\depth^\text{max}$.
