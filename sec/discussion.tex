% !TEX root = ../hmodel.tex
\section{Discussion}
\label{sec:discussion}
\begin{figure}[t]
\centering
\begin{overpic} 
[width=\linewidth]
% [width=\linewidth,grid,tics=10]
{\currfiledir/item.pdf}
% \put(40,25){\todo{\Large DRAFT}}
% \put(0,-3){\todo{\currfiledir}}
\end{overpic}
\caption{
% 
Our dataset contains a number of wide range of motions that spans the entire literature on hand tracking. We identify three main axes of variations by analyzing recent hand-tracking papers, and produce a dataset that samples this space. \TODO{export the same image, but **without** text, also fill the entire width so to make the models bigger. Also you color code, but then you use two axes with the **same** color? Then what's the point of color-coding? Also, you need to add numbers to each of them! (needed for a figure later)}
% 
}
\label{fig:motiontypes}
\end{figure}

\AT{Sentence from CONCERTO: ``The correspondences now contribute only a small computational cost to each iteration, comparing very favourably with ICP, where a 
closest-point operation must be performed on each iteration for each data point.'' For convtrack this is not a problem because it's very cheap, yet we can get a very good fitting model}

\AT{Something about the fact that rather than mentioning thresholds and values and gradients we just provide source code for it?}

% The core investigation in this paper is whether convolutional surfaces are a viable alternative for realtime hand tracking with single-view RBGD sensors. 
% Our hypothesis is that the increased geometry accuracy compared with cylinder models allows for a higher tracking precision without impairing the computational performance of the tracking algorithm.

\TODO{mention no need to re-init $\gg$FPS?}.

\Anastasia{Also, I think it is important to mention that in all the sequences the depth data is downsampled 4 times. Otherwise, the reviewers might say that our tracking looks better because the hand is very close to the sensor. It is close that that it can be seen better}
