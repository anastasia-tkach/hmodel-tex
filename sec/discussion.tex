% !TEX root = ../hmodel.tex
\section{Discussion}
\label{sec:discussion}
\begin{figure}[t!]
\centering
\begin{overpic} 
[width=\linewidth]
% [width=\linewidth,grid,tics=10]
{\currfiledir/item.pdf}
% \put(40,25){\todo{\Large DRAFT}}
% \put(0,-3){\todo{\currfiledir}}
\end{overpic}
\caption{
% 
Our dataset contains a number of wide range of motions that spans the entire literature on hand tracking. We identify three main axes of variations by analyzing recent hand-tracking papers, and produce a dataset that samples this space. \TODO{mention they come from teaser sequence, and position in video, 150 microseconds 9 frames}
% 
}
\label{fig:motiontypes}
\end{figure}

% \paragraph{Calibration}
% \Anastasia{The calibration is still not perfect, there is several components to it, centers position, initial transformations, joint limits and PCA (initial transformations, joint limits and PCA seem to be similar for all humans, but in a perfect world we could try to optimize for all of them).}
% \Anastasia{Maybe, this goes too far, but we might say that our model calibration has so little degrees of freedom compared to a triangles mesh, that it does not depend on a prior as heavily as triangular mesh-based systems. Those systems cannot work without a prior - too many degrees of freedom. And a prior means acquiring a database of hands. Not any research team can accomplish that.}

\AT{Perhaps, we should perhaps compare volumetric v.s. surface representations here? How triangle meshes need substantial regularization, the specification of control skeleton, skinning, collision elements, etc... while our models achieve very good visual fidelity by just positions and radii. Mark, if you can see a nice way of saying this, I'd really appreciate it.}

\paragraph{Generative tracker}
% \Anastasia{We can stress again that we do not have re-initialization not because it is somehow fundamentally impossible in our method, but because we want to demonstrate how good our tracking is. It basically only brakes in limitation cases (fast motion, other objects, rotating thumb, out of sensor range.) For that we could totally add some powerful re-initialization to make the system even more robust.} \Anastasia{Also, I think we could mention that our system can be optimized to 120 FPS maybe? The tracking quality degrades with speed of motion, that is with number of frames per some ``quantity of motion''. With faster frame rate the system will definitely get better.} \AT{not fully true, faster framerate means more noise!}
We would like to stress that our real-time tracking algorithm is \emph{purely} generative, yet in this paper we demonstrated this yielding unprecedented levels of robustness to tracking failure. Currently the only sequences that incur in loss-of-tracking are the ones reported in our limitations \VideoLimit{}. We believe this is due to the ability to optimize at a constant 60Hz rate, and to the quality of the calibrated model. Discriminative algorithms could still be necessary to compensate for situations where the hand re-appears from complete occlusions, but we believe their role for real-time tracking will diminish as RGBD sensors will start offering 120Hz imaging.  


\paragraph{Downsampling}
% \Anastasia{Also, I think it is important to mention that in all the sequences the depth data is downsampled 4 times. Otherwise, the reviewers might say that our tracking looks better because the hand is very close to the sensor. It is close than that it can be seen better}
Although the \realsense{} sensor is a short range camera, notice in this work we have downsampled the depth image 4 times to \todo{QVGA} format with a median filter, giving an average of \todo{10000} pixels/frame for optimization; note this is approximatively the number of samples found on a hand in long-range cameras. The recent work of \cite{taylor2016concerto} reports a total of 192 pixels/frame, therefore enabling multi-CPU performance without loss of tracking precision. Inspired by this work, we have experimented with further downsampling and reached analogous conclusions. However, the computational bottleneck of the \emph{htrack} system we extend lies in the overhead caused by render/compute context switching. While this is currently an issue, we would like to note that this problem will disappear with the adoption of \emph{compute shaders}~(OpenGL 4.3).

\paragraph{Reproducibility}
% 
The weights of energy terms used in tracking and calibration optimizations have been identified by manually tweaking the runtime until our tracker reached the desired performance level. 
We do not believe reporting the values for such weights here to be particularly meaningful. 
\MP{Do we use the same weights always or does every sequence have to be tweaked individually? This is could be a statement that some people don't like. We could just list the values even though they are meaningless...} 
However, we adhere to the \textsc{lgpl} requirements from \emph{htrack}, and (upon acceptance) release our source code with parameters, datasets, as well as the specification of the energy gradients required for optimization. This will ensure full reproducibility of our work.

