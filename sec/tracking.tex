\section{Tracking}

\paragraph{Element Projection}
\TODO{Anastasia, please add and describe \Figure{corresp}, that is how do we compute correspondences to a single convolution element}.

\paragraph{Element-Wise Correspondences}
Our correspondence search leverages the structure of \Eq{convsurf}, by decomposing the surface in several elementary convolution elements $\element$ (\todo{a decomposition of the shape in pill and wedge primitives}) and their associated implicit functions $\implicit_e$. Given a point $\point$ in space, the implicit function of the whole surface can be written by evaluating the expression:
\begin{equation}
\implicit_\surface(\point) = \argmin_{e=1 \dots E} \implicit_e(\point)
\label{eq:piecewise}
\end{equation}
Therefore, given a query point $\point$, we can first compute the closest-points $\footpoint_e = \proj_{\element}(\point)$ to each element independently; within this set, the closest-point projection to the full model $\footpoint = \proj_\surface(\point)$ is the one with the smallest associated implicit function value $\implicit_e(\point)$.
\TODO{should we mention how long this takes?}

\begin{figure}[t]
\centering
\begin{overpic} 
[width=\linewidth]
% [width=\linewidth,grid,tics=10]
{fig/corresp/item.pdf}
% \put(10,10){\todo{\Large Overlay Text}}
\end{overpic}
\caption{(left) Projecting point on a pill. (right) Computing tangent planes for a wedge. }
\label{fig:corresp}
\end{figure}
\paragraph{Frontal Correspondences}
In monocular acquisition, an oracle registration algorithm aligns the portion of the model that is \emph{visible} from sensor viewpoint to the available data. Therefore, when computing ICP's closest-point correspondences, only the portion of the model currently visible by the camera should be considered~\cite{tagliasacchi2015robust}. Given the camera direction $\camdir$, we can test whether the retrieved footpoint $\footpoint$ is back-facing by testing the sign of $\camdir \cdot \mathcal{N}_\surface(\footpoint)$, where the second term is the object's normal at $\footpoint$. As illustrated in \Figure{?}, whenever this test fails, there are additional candidates for closest point that must be checked . The first with respect to the silhouette of the model, 

The \emph{silhouette} of an object is a (3D) curve separating front from back-facing portions of a shape. If we restrict correspondence search to the frontal portion of the model, naturally many correspondences will be mapped to such curve. 

\FINISH

% In convolution surfaces correspondence computation takes place in four steps:
% \todo{If our convolution model would be composed of a single skeletal element, the computation of correspondences would be as visualized in~\Figure{corresp}.}

% \begin{figure}[b]
\centering
\begin{overpic} 
[width=\linewidth]
{fig/silhouette/item.png}
\end{overpic}
\caption{
% 
% 
(left) The silhouette of the model computed by projecting the model in the camera plane (\todo{fingers outline is computed separately, here an entire model outline is shown for illustration purposes}). (right) The silhouette curves, marked in pink, are re-projected in 3D. 
\AT{why in the image in the left it's only the image-space silhouette with a pink boundary, while on the right you can also find the silhouette in the interior?}
\Anastasia{As I mentioned, for illustration purposes, to show to the outline is always outside of the model. Probably it is a bad idea, I should just display the same outline as on the right, because that is what I actually compute.}
} 
\label{fig:silhouette}
\end{figure}

\begin{algorithm}
\caption{Correspondences computation}
\begin{algorithmic}[1]
    \For {$\text{each } p$}
    	 \State \text{compute model projection } $q_m$
    	 \State \text{replace or discard if $q_m$ is back-facing}
    	 \State \text{compute outline projection } $q_o$
         \State $q=(\|{p - q_m}\|_2^2 < \|{p - q_o}\|_2) ? q_m : q_o$
    \EndFor
\end{algorithmic}
\label{alg:correspondences}
\end{algorithm}

% \textbf{Computing model projection.}
% taking the minimum helps to get the projection on the model surface when the data point $p$ is inside of the model \AT{how? I am not sure I understand, please quickly sketch a figure!)}.

\textbf{Discarding or replacing back-facing projections.}
\begin{itemize}
	\item Convolution segment: the closest front-facing point is on the model outline, thus set the current point $q_m$ to $\infty$.
	\item Convolution triangle: the closest front-facing point is either on the model outline or on a front-facing face of the convolution triangle, thus replace $q_m$ by the closest front-facing face projection.
\end{itemize}

\textbf{Computing outline.}
For this computation we assume that projection is orthographic and that camera direction coincides with axis $Z$. We shift all the model spheres to have zero coordinate at axis $Z$ and compute an outline of the cross-section of the model with the $XY$ plane. This outline is computed by finding the upper left point of the cross-section and traversing the graph of line and circle segments in counters-clockwise direction (see Figure \ref{fig:silhcorr}). To obtain the 3D outline we shift the model spheres with attached 2D outline back to their original positions.
\AT{What's the difference between the silhouette here and the one computed by raytracing/rasterizing the model on the GPU?}




\textbf{Computing outline projection.}
The model outline is represented as a sequence of line and circle segments. To compute an outline projection $q_o$ we compute a projection on each element of the outline and select the closest to the data point $p$.

\begin{figure}[t]
\centering
\begin{overpic} 
[width=\linewidth]
% [width=\linewidth,grid,tics=10]
{fig/visibility/item.pdf}
% \put(10,10){\todo{\Large Overlay Text}}
\end{overpic}
\caption{In a monocular RGBD sensor only the front-facing part of the model should be registered to the data. Here the camera is observing (left to right) two 2D convolution elements and the visible parts of the model are color-coded. Correspondences whose normal point away from the camera ought to be discarded, and replaced by the closest amongst silhouette correspondences or front-facing portions of wedges.}
\label{fig:visibility}
\end{figure}