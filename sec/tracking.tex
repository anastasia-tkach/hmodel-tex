\section{Tracking}

\begin{figure}[t]
\centering
\begin{overpic} 
[width=\linewidth]
% [width=\linewidth,grid,tics=10]
{fig/corresp/item.pdf}
% \put(10,10){\todo{\Large Overlay Text}}
\end{overpic}
\caption{(left) Projecting point on a pill. (right) Computing tangent planes for a wedge. }
\label{fig:corresp}
\end{figure}


\paragraph{Element Correspondences}

%%%%%%%%%%%%%%%%%%%%%%%%%%%%%%%%%%%%%%%%%%%%%%%%%%%%%%%%%%%%%%%%%%%%%%%%

\textbf{Convolution segment}

A pill is defined by two spheres $S_1 = \{c_1, r_1\}$ and $S_2 = \{c_2, r_2\}$. Assume that $c_1 > c_2$. 

Let us compute the position of $q$  - projection of the data point $p$ on the segment $\{c_1, c_2\}$.  To do this we compute ``skewed`` projection of $p$ on the skeleton segment $c_1 c_2$. This means that we compute the projection $s$ on a segment not along a line orthogonal to the segment, but along a line orthogonal to conic surface of the pill.
$u = c_2 - c_1$,  
$ v = p - c_1$, 
$\alpha = \frac{u^T v }{u^T  u}$, 
$ t = c_1 + \alpha u $,
where $t$ is an orthogonal projection of the point on a segment. The next step is to compute $s$.
Find $\omega$ from Pythagorean theorem 
$\omega^2 = u^T u - (r_1 - r_2)^2$,
 and $\delta$ from similarity of triangles
$\frac{\delta}{\Vert p - t  \Vert} =  \frac{r_1 - r_2}{\omega}$. The skewed projection $s$ is found by shifting $t$ by $\delta$ along the axis of the segment $s = t - \delta  \frac{u} {\Vert u \Vert}$.


\begin{itemize}
\item If the skewed projection $s$ lies on the segment (that is between points $c_1$ and $c_2$), than the data point $p$ projects on the conic surface of the 	pill. Find $\gamma$ from similarity of triangles:  $\frac{\gamma}{ {\Vert c_2 - s}\Vert} = \frac{r_1 - r_2} {\Vert u \Vert}$. Find $q$ by shifting $s$ along the segment $sp$ by $\gamma + r_2$.
\item If the skewed projection lies outside of the segment and, say, closer to $c_1$ than the data point $p$ projects on the sphere  $S_1 = \{c_1, r_1\}$, that is  $q = c_1 + \frac{r_1  (p - c_1)}{\Vert p - c_1 \Vert}$. 
\end{itemize}


%%%%%%%%%%%%%%%%%%%%%%%%%%%%%%%%%%%%%%%%%%%%%%%%%%%%%%%%%%%%%%%%%%%%%%%%
\textbf{Convolution triangle}

A wedge is defined by three spheres $S_1 = \{c_1, r_1\}$, $S_2 = \{c_2, r_2\}$ and $S_3 = \{c_3, r_3\}$. 

Before computing the projection, we need to compute the vertices $u_1, u_2, u_3$ and $v_1, v_2, v_3$ of the two triangles tangent to the spheres that form the faces of the wedge. For tracking this computation is done only once for each wedge at initialization.

Denote the apex of the cone tangent to the spheres $S_1$ and $S_2$ as $z_{12}$. Find $z_{12}$ from similarity of triangles $z_{12} = c_1 +\frac{r_1 (c_2 - c_1)}{r_1 - r_2}$. In the same way $z_{13} = c_1 +\frac{r_1 (c_3 - c_1)}{r_1 - r_3}$.

The vector $l = \frac{z_{12} - z_{13}} {\Vert z_{12} - z_{13} \Vert}$ is the direction vector of the line, that contains the apices of the tangent cones.

The intersection point of the plane orthogonal to $l$ and going through the point $c_1$ with the line going through $z_{12}$ and $z_{13}$ is found as
$z = z_{12} + ((c_1 - z_{12})^T l )l$.
The sine and cosine of the angle $\beta$ in triangle $\{c_1, z, v_1\}$ are given by 
$\sin(\beta) = \frac{r_1}{\eta}$ 
and 
$\cos(\beta) = \frac{\nu}{\eta}$,
where $\eta = \Vert c_1 - z \Vert $ and $\nu = \sqrt{\eta^2 -  r_1^2}$. 

Denote the tangent point of the sphere $S_1$ as $v_1$. The direction vector $g$ of the line $\{c_1, v_1\}$ can be found by rotating the direction vector $f$ of the line 
$\{c_1, z\}$ by angle $\beta$ around the axis $l$.

\begin{align*}
	& h = \frac{l \times f}{\Vert l \times f \Vert}, \\
	& g = \sin(\beta) h + \cos(\beta) f.	
\end{align*}

The tangent point $v_1$ is found as $v_1 = z + \nu  g$. Having the normal vector of the tangent plane $n = \frac{v_1 - c_1}{\Vert v_1 - c_1 \Vert}$, we can find the tangent points of the spheres $S_2$ and $S_3$ as $v_2 = c_2  + r_2 n$ and $v_3 = c_3 + r_3 n$.
The second tangent plane $\{u_1, u_2, u_3$\} is found by rotating the vector $f$ around the axis $l$ by an angle $-\beta$.

As in the case with pill, we compute skewed projections $s_1$ and $s_2$ of the point $p$ on the skeleton  - the triangle $c_1 c_2 c_3$ along the normals to the triangles $v_1 v_2 v_3$ and $u_1 u_2 u_3$.

\begin{itemize}
\item If one or both projections lie inside of the triangle $c_1 c_2 c_3$ than the point $p$ projects on the face of the wedge. (In case if it projects on both faces, just pick the closest projection.) \Anastasia{For tracking I only consider projections to front-facing faces.}
\item If none of the projections line inside of the triangle, but, say, $s_1$ lies on the edge $c_1 c_2$, than $p$ projects on the conic surface tangent to the spheres $S_1$ and $S_2$. 
\item If, say, $s_1$ projects to the vertex $c_1$, than $p$ projects to the sphere $S_1$.
\end{itemize}

\Anastasia{Actually in the code I am projecting the point to all the conic surfaces of the wedge, not just to one, so I never experimentally verified that this should be correct in all the cases. But I cannot come up with a counter-example. I should test it and optimize the code.}


\paragraph{Element-Wise Correspondences}
Our correspondence search leverages the structure of \Eq{convsurf}, by decomposing the surface in several elementary convolution elements $\element$ (a decomposition of the shape in \todo{pill} and \todo{wedge} implicit primitives) and their associated implicit functions $\implicit_e$. Given a point $\point$ in space, the implicit function of the whole surface can be written by evaluating the expression:
\begin{equation}
\implicit_\surface(\point) = \argmin_{e=1 \dots E} \implicit_e(\point)
\label{eq:piecewise}
\end{equation}
Therefore, given a query point $\point$, we can first compute the closest-points $\footpoint_e = \proj_{\element}(\point)$ to each element independently; within this set, the closest-point projection to the full model $\footpoint = \proj_\surface(\point)$ is the one with the smallest associated implicit function value $\implicit_e(\point)$.
\TODO{should we mention how long this takes?}

\paragraph{Frontal Correspondences}
In monocular acquisition, an oracle registration algorithm aligns the portion of the model that is \emph{visible} from sensor viewpoint to the available data. Hence, when computing ICP's closest-point correspondences, only the portion of the model currently visible by the camera should be considered~\cite{tagliasacchi2015robust}. Given the camera direction $\camdir$, we can test whether the retrieved footpoint $\footpoint$ is back-facing by testing the sign of $\camdir \cdot \mathcal{N}_\surface(\footpoint)$, where the second term is the object's normal at $\footpoint$. As illustrated in 2D in \Figure{visibility}, whenever this test fails, there are additional candidates for closest point that must be checked: (1)~the closest-point on the silhouette of the model (e.g. $\point_{2,4,5,7}$), and (2)~the front facing planar portions of convolution elements (e.g. $\point_{3}$). These additional correspondences for the query point are computed, and the one closest to $\point$ becomes our front-facing footpoint $\footpoint$.
\begin{figure}[t]
\centering
\begin{overpic} 
[width=\linewidth]
% [width=\linewidth,grid,tics=10]
{fig/visibility/item.pdf}
% \put(10,10){\todo{\Large Overlay Text}}
\end{overpic}
\caption{In a monocular RGBD sensor only the front-facing part of the model should be registered to the data. Here the camera is observing (left to right) two 2D convolution elements and the visible parts of the model are color-coded. Correspondences whose normal point away from the camera ought to be discarded, and replaced by the closest amongst silhouette correspondences or front-facing portions of wedges.}
\label{fig:visibility}
\end{figure}


\begin{figure}[b]
\centering
\begin{overpic} 
[width=\linewidth]
{fig/silhouette/item.png}
\end{overpic}
\caption{
% 
% 
(left) The silhouette of the model computed by projecting the model in the camera plane (\todo{fingers outline is computed separately, here an entire model outline is shown for illustration purposes}). (right) The silhouette curves, marked in pink, are re-projected in 3D. 
\AT{why in the image in the left it's only the image-space silhouette with a pink boundary, while on the right you can also find the silhouette in the interior?}
\Anastasia{As I mentioned, for illustration purposes, to show to the outline is always outside of the model. Probably it is a bad idea, I should just display the same outline as on the right, because that is what I actually compute.}
} 
\label{fig:silhouette}
\end{figure}
\paragraph{Silhouette Computation}
The \emph{silhouette} is a (3D) curve separating front-facing from back-facing portions of a shape. \Anastasia{Normally by silhouette people mean the outline of the shape and everything inside of the outline. "A silhouette is the image of a person, animal, object or scene represented as a solid shape of a single color, usually black, its edges matching the outline of the subject (Wikipedia)". Also, silhouette is always in 2D, like in our silhouette energy} In our visibility-aware search, many points will be mapped to this curve. Similarly to \cite{tagliasacchi2015robust}, to compute the silhouette we approximate the perspective camera of the sensor with an orthographic one. Under this assumption finding the (3D) silhouette $\partial \surface$ becomes much simpler. 

\textcolor{DarkOrchid} {
We first shift all the model centers to have zero coordinate at the axis Z (which coincides with camera axis). We compute the cross-section of the model with XY plane. This cross-section consists of  circles and line segments (see \Figure{silhouette}, left). Now we need to find a set of circle segments and line segments that are on the outline of the cross-section. This is done by computing intersection and tangency points of every circle with every other circle and every line segment. The resulting structure can be thought of as a graph with intersection and tangency points as vertices and circle and line segments as edges. We traverse this graph starting from the upper left or any other point that is guaranteed to be on the outline. From every vertex we follow the edge with the next polar angle from the one that we came from (for the circle the polar angle is computed for a tangent at that point). This way  we always stay on the outline and never go inside of the cross-section.
}

\textcolor{DarkOrchid} {
Note that if a finger is in front of the palm, we still want to have its outline for the benefit of the correspondences that are back-facing to the finger.
Therefore we separately compute the outline for the palm and for the fingers and merge them together afterwards. The merging is done by removing the part of the finger outline that is inside of the palm outline and part of the palm outline that is inside of the finger outline. The fingers outline only modified if it is the outline of the base finger segment, because it is OK for the finger tip outline to be inside of the palm outline.
}

\textcolor{DarkOrchid} {
Once the 2D silhouette has been computed, it can be re-projected in 3D; see \Figure{silhouette}.
}

\AT{do you have to pass the silhouette back to CUDA to compute correspondences?}
\Anastasia{the outline is computed on CPU, the outline projections are computed on GPU. My function for computing  correspondences looks exactly like Algorithm 1 that I have wrote.}

\FINISH
