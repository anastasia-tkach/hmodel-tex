\section{Tracking}

\begin{figure}[t]
\centering
\begin{overpic} 
[width=\linewidth]
% [width=\linewidth,grid,tics=10]
{fig/corresp/item.pdf}
% \put(10,10){\todo{\Large Overlay Text}}
\end{overpic}
\caption{(left) Projecting point on a pill. (right) Computing tangent planes for a wedge. }
\label{fig:corresp}
\end{figure}
\paragraph{Element Correspondences}
\TODO{Anastasia, please add and describe \Figure{corresp}, that is how do we compute correspondences to a single convolution element}.

\paragraph{Element-Wise Correspondences}
Our correspondence search leverages the structure of \Eq{convsurf}, by decomposing the surface in several elementary convolution elements $\element$ (a decomposition of the shape in \todo{pill} and \todo{wedge} implicit primitives) and their associated implicit functions $\implicit_e$. Given a point $\point$ in space, the implicit function of the whole surface can be written by evaluating the expression:
\begin{equation}
\implicit_\surface(\point) = \argmin_{e=1 \dots E} \implicit_e(\point)
\label{eq:piecewise}
\end{equation}
Therefore, given a query point $\point$, we can first compute the closest-points $\footpoint_e = \proj_{\element}(\point)$ to each element independently; within this set, the closest-point projection to the full model $\footpoint = \proj_\surface(\point)$ is the one with the smallest associated implicit function value $\implicit_e(\point)$.
\TODO{should we mention how long this takes?}

\paragraph{Frontal Correspondences}
In monocular acquisition, an oracle registration algorithm aligns the portion of the model that is \emph{visible} from sensor viewpoint to the available data. Hence, when computing ICP's closest-point correspondences, only the portion of the model currently visible by the camera should be considered~\cite{tagliasacchi2015robust}. Given the camera direction $\camdir$, we can test whether the retrieved footpoint $\footpoint$ is back-facing by testing the sign of $\camdir \cdot \mathcal{N}_\surface(\footpoint)$, where the second term is the object's normal at $\footpoint$. As illustrated in 2D in \Figure{visibility}, whenever this test fails, there are additional candidates for closest point that must be checked: (1)~the closest-point on the silhouette of the model (e.g. $\point_{2,4,5,7}$), and (2)~the front facing planar portions of convolution elements (e.g. $\point_{3}$). These additional correspondences for the query point are computed, and the one closest to $\point$ becomes our front-facing footpoint $\footpoint$.
\begin{figure}[t]
\centering
\begin{overpic} 
[width=\linewidth]
% [width=\linewidth,grid,tics=10]
{fig/visibility/item.pdf}
% \put(10,10){\todo{\Large Overlay Text}}
\end{overpic}
\caption{In a monocular RGBD sensor only the front-facing part of the model should be registered to the data. Here the camera is observing (left to right) two 2D convolution elements and the visible parts of the model are color-coded. Correspondences whose normal point away from the camera ought to be discarded, and replaced by the closest amongst silhouette correspondences or front-facing portions of wedges.}
\label{fig:visibility}
\end{figure}


\begin{figure}[b]
\centering
\begin{overpic} 
[width=\linewidth]
{fig/silhouette/item.png}
\end{overpic}
\caption{
% 
% 
(left) The silhouette of the model computed by projecting the model in the camera plane (\todo{fingers outline is computed separately, here an entire model outline is shown for illustration purposes}). (right) The silhouette curves, marked in pink, are re-projected in 3D. 
\AT{why in the image in the left it's only the image-space silhouette with a pink boundary, while on the right you can also find the silhouette in the interior?}
\Anastasia{As I mentioned, for illustration purposes, to show to the outline is always outside of the model. Probably it is a bad idea, I should just display the same outline as on the right, because that is what I actually compute.}
} 
\label{fig:silhouette}
\end{figure}
\paragraph{Silhouette Computation}
The \emph{silhouette} is a (3D) curve separating front-facing from back-facing portions of a shape. In our visibility-aware search, many points will be mapped to this curve. Similarly to \cite{tagliasacchi2015robust}, to compute the silhouette we approximate the perspective camera of the sensor with an orthographic one. Under this assumption finding the (3D) silhouette $\partial \surface$ becomes much simpler. We first project the 3D convolution model onto the 2D camera plane, and then traverse the graph of line and circle segments in counters-clockwise direction. \AT{this is not very clear to me, needs to be explained better} Once the 2D silhouette has been computed, it can be re-projected in 3D; see \Figure{silhouette}.

\AT{do you have to pass the silhouette back to CUDA to compute correspondences?}

\FINISH
