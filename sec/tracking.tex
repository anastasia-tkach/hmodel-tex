\section{Tracking}

\paragraph{Correspondences}
In monocular acquisition, an oracle registration algorithm would align only the portion of the model that is \emph{visible} from sensor point of view~\cite{tagliasacchi2015robust}. Therefore, when computing ICP correspondences, only the portion of the model currently visible by the camera should be considered.

\todo{{----------- ANDREA - FINISHED EDITING HERE -----------}}


% In convolution surfaces correspondence computation takes place in four steps:

% \todo{If our convolution model would be composed of a single skeletal element, the computation of correspondences would be as visualized in~\Figure{corresp}.}

\begin{figure}[t]
\centering
\begin{overpic} 
[width=\linewidth]
% [width=\linewidth,grid,tics=10]
{fig/corresp/item.pdf}
% \put(10,10){\todo{\Large Overlay Text}}
\end{overpic}
\caption{
% 
% 
The computation of closest point correspondences on pill (left) and wedge (right) convolution elements can be performed by tracing a ray along the normal of the circles (resp. sphere) tangent line (resp. plane). \TODO{how to subscript in illustrator?} \MP{I usually prefer overpic and use latex. It's a bit painful, but usually looks much nicer}
%
%
}
\label{fig:corresp}
\end{figure}

\subsection*{Correspondences Computation \todo{(OLD)}}

\begin{algorithm}
\caption{Correspondences computation \todo{(rename $p$ to $\point$)}}
\begin{algorithmic}[1]
    \For {$\text{each } p$}
    	 \State \text{compute model projection } $q_m$
    	 \State \text{replace or discard if $q_m$ is back-facing}
    	 \State \text{compute outline projection } $q_o$
         \State $q=(\|{p - q_m}\|_2^2 < \|{p - q_o}\|_2) ? q_m : q_o$
    \EndFor
\end{algorithmic}
\label{alg:correspondences}
\end{algorithm}

\textbf{Computing model projection.}
For each each ``block'' $b$, compute $q_b = \proj_b(\point)$, then the
projection over the whole model $q_m = \proj_\surface(\point)$ is chosen as: 
\begin{equation*}
	q_m = \argmin_{q_b = \{ q_1, \dots, q_\numblocks \}} \phi_{q_b}(p) 
\end{equation*}

taking the minimum helps to get the projection on the model surface when the data point $p$ is inside of the model \AT{how? I am not sure I understand, please quickly sketch a figure!)}.

\textbf{Discarding or replacing back-facing projections.}
The resulting projection $q_m$ can be on back-facing side of the model, that is $c^{T} n > 0$, where $c$ is camera ray and $n$ is model normal at the point $q_m$.
The next step depends on the block type. 
\begin{itemize}
	\item Convolution segment: the closest front-facing point is on the model outline \Anastasia{do I need to prove this? I could make a distance field picture in 2D}, thus set the current point $q_m$ to $\infty$.
	\item Convolution triangle: the closest front-facing point is either on the model outline or on a front-facing face of the convolution triangle, thus replace $q_m$ by the closest front-facing face projection.
\end{itemize}

\textbf{Computing outline.}
For this computation we assume that projection is orthographic and that camera direction coincides with axis $Z$. We shift all the model spheres to have zero coordinate at axis $Z$ and compute an outline of the cross-section of the model with the $XY$ plane. This outline is computed by finding the upper left point of the cross-section and traversing the graph of line and circle segments in counters-clockwise direction (see Figure \ref{fig:silhcorr}). To obtain the 3D outline we shift the model spheres with attached 2D outline back to their original positions.
\AT{What's the difference between the silhouette here and the one computed by raytracing/rasterizing the model on the GPU?}


\begin{figure}[t!]
\centering
\begin{overpic} 
[width=\linewidth]
% [width=\linewidth,grid,tics=10]
{fig/topology/item.pdf}
%{fig/topology/topology}
% \put(10,10){\todo{fig:topology}}
\end{overpic}
\caption{
% 
% 
(left) The skeleton $\skeleton$ parametrizes the convolution surface providing a radii value property on vertices. 
In our template, articulated components are shown in {\color{darkgreen} dark green} while flexible components in {\color{purple}purple}.
% 
(right) Calibration instantiates our template by adjusting the skeletal vertex positions and radii. 
% 
% 
}
\label{fig:topology}
\end{figure}

\textbf{Computing outline projection.}
The model outline is represented as a sequence of line and circle segments. To compute an outline projection $q_o$ we compute a projection on each element of the outline and select the closest to the data point $p$.