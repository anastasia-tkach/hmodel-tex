
\paragraph{Element Correspondences}

%%%%%%%%%%%%%%%%%%%%%%%%%%%%%%%%%%%%%%%%%%%%%%%%%%%%%%%%%%%%%%%%%%%%%%%%

\textbf{Convolution segment}

A pill is defined by two spheres $S_1 = \{c_1, r_1\}$ and $S_2 = \{c_2, r_2\}$. Assume that $c_1 > c_2$. 

Let us compute the position of $q$  - projection of the data point $p$ on the segment $\{c_1, c_2\}$.  To do this we compute ``skewed`` projection of $p$ on the skeleton segment $c_1 c_2$. This means that we compute the projection $s$ on a segment not along a line orthogonal to the segment, but along a line orthogonal to conic surface of the pill.
$u = c_2 - c_1$,  
$ v = p - c_1$, 
$\alpha = \frac{u^T v }{u^T  u}$, 
$ t = c_1 + \alpha u $,
where $t$ is an orthogonal projection of the point on a segment. The next step is to compute $s$.
Find $\omega$ from Pythagorean theorem 
$\omega^2 = u^T u - (r_1 - r_2)^2$,
 and $\delta$ from similarity of triangles
$\frac{\delta}{\Vert p - t  \Vert} =  \frac{r_1 - r_2}{\omega}$. The skewed projection $s$ is found by shifting $t$ by $\delta$ along the axis of the segment $s = t - \delta  \frac{u} {\Vert u \Vert}$.


\begin{itemize}
\item If the skewed projection $s$ lies on the segment (that is between points $c_1$ and $c_2$), than the data point $p$ projects on the conic surface of the 	pill. Find $\gamma$ from similarity of triangles:  $\frac{\gamma}{ {\Vert c_2 - s}\Vert} = \frac{r_1 - r_2} {\Vert u \Vert}$. Find $q$ by shifting $s$ along the segment $sp$ by $\gamma + r_2$.
\item If the skewed projection lies outside of the segment and, say, closer to $c_1$ than the data point $p$ projects on the sphere  $S_1 = \{c_1, r_1\}$, that is  $q = c_1 + \frac{r_1  (p - c_1)}{\Vert p - c_1 \Vert}$. 
\end{itemize}


%%%%%%%%%%%%%%%%%%%%%%%%%%%%%%%%%%%%%%%%%%%%%%%%%%%%%%%%%%%%%%%%%%%%%%%%
\textbf{Convolution triangle}

A wedge is defined by three spheres $S_1 = \{c_1, r_1\}$, $S_2 = \{c_2, r_2\}$ and $S_3 = \{c_3, r_3\}$. 

Before computing the projection, we need to compute the vertices $u_1, u_2, u_3$ and $v_1, v_2, v_3$ of the two triangles tangent to the spheres that form the faces of the wedge. For tracking this computation is done only once for each wedge at initialization.

Denote the apex of the cone tangent to the spheres $S_1$ and $S_2$ as $z_{12}$. Find $z_{12}$ from similarity of triangles $z_{12} = c_1 +\frac{r_1 (c_2 - c_1)}{r_1 - r_2}$. In the same way $z_{13} = c_1 +\frac{r_1 (c_3 - c_1)}{r_1 - r_3}$.

The vector $l = \frac{z_{12} - z_{13}} {\Vert z_{12} - z_{13} \Vert}$ is the direction vector of the line, that contains the apices of the tangent cones.

The intersection point of the plane orthogonal to $l$ and going through the point $c_1$ with the line going through $z_{12}$ and $z_{13}$ is found as
$z = z_{12} + ((c_1 - z_{12})^T l )l$.
The sine and cosine of the angle $\beta$ in triangle $\{c_1, z, v_1\}$ are given by 
$\sin(\beta) = \frac{r_1}{\eta}$ 
and 
$\cos(\beta) = \frac{\nu}{\eta}$,
where $\eta = \Vert c_1 - z \Vert $ and $\nu = \sqrt{\eta^2 -  r_1^2}$. 

Denote the tangent point of the sphere $S_1$ as $v_1$. The direction vector $g$ of the line $\{c_1, v_1\}$ can be found by rotating the direction vector $f$ of the line 
$\{c_1, z\}$ by angle $\beta$ around the axis $l$.

\begin{align*}
	& h = \frac{l \times f}{\Vert l \times f \Vert}, \\
	& g = \sin(\beta) h + \cos(\beta) f.	
\end{align*}

The tangent point $v_1$ is found as $v_1 = z + \nu  g$. Having the normal vector of the tangent plane $n = \frac{v_1 - c_1}{\Vert v_1 - c_1 \Vert}$, we can find the tangent points of the spheres $S_2$ and $S_3$ as $v_2 = c_2  + r_2 n$ and $v_3 = c_3 + r_3 n$.
The second tangent plane $\{u_1, u_2, u_3$\} is found by rotating the vector $f$ around the axis $l$ by an angle $-\beta$.

As in the case with pill, we compute skewed projections $s_1$ and $s_2$ of the point $p$ on the skeleton  - the triangle $c_1 c_2 c_3$ along the normals to the triangles $v_1 v_2 v_3$ and $u_1 u_2 u_3$.

\begin{itemize}
\item If one or both projections lie inside of the triangle $c_1 c_2 c_3$ than the point $p$ projects on the face of the wedge. (In case if it projects on both faces, just pick the closest projection.) \Anastasia{For tracking I only consider projections to front-facing faces.}
\item If none of the projections line inside of the triangle, but, say, $s_1$ lies on the edge $c_1 c_2$, than $p$ projects on the conic surface tangent to the spheres $S_1$ and $S_2$. 
\item If, say, $s_1$ projects to the vertex $c_1$, than $p$ projects to the sphere $S_1$.
\end{itemize}

\Anastasia{Actually in the code I am projecting the point to all the conic surfaces of the wedge, not just to one, so I never experimentally verified that this should be correct in all the cases. But I cannot come up with a counter-example. I should test it and optimize the code.}
