\begin{figure}[t!]
\centering
\begin{overpic} 
[width=\linewidth]
% [width=\linewidth,grid,tics=10]
{fig/visibility/item.pdf}
\put(1,34){\small{$\point_1$}}
\put(12,27){\small{$\footpoint_1$}}
\put(45,28){\small{$\point_2$}}
\put(23,33){\small{$\footpoint_2$}}
\put(27,15){\small{$\point_3$}}
\put(23,1.5){\small{$\footpoint_3$}}
% 
\put(56,33){\small{$\point_4$}}
\put(67,31){\small{$\footpoint_4$}}
\put(85,30){\small{$\point_5$}}
\put(63,37){\small{$\footpoint_5$}}
\put(96,56){\small{$\point_6$}}
\put(78,56.5){\small{$\footpoint_6$}}
\put(85,16.5){\small{$\point_7$}}
\put(78,1){\small{$\footpoint_7$}}
\put(51,8){\small{$\point_8$}}
\put(66,13){\small{$\footpoint_8$}}
% 
\put(8,56){query point}
\put(8,52){front-facing correspondence}
\put(8,48){back-facing  correspondence}
\put(8,44){camera view direction}

\end{overpic}
\caption{
% 
% 
In \todo{monocular acquisition} only the front-facing part of the model should be registered to the data. Here the camera is observing (left to right) two elements and the occluded parts of the model are marked. Correspondences whose normals point away from the camera \todo{are} discarded, and replaced by the closest amongst silhouette correspondences or front-facing portions of wedges.
% 
% 
}
\label{fig:visibility}
\end{figure}
