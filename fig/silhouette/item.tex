\begin{figure}[b]
\centering
\begin{overpic} 
[width=\linewidth]
{fig/silhouette/item.png}
\end{overpic}
\caption{
% 
% 
(left) The silhouette of the model computed by projecting the model in the camera plane (\todo{fingers outline is computed separately, here an entire model outline is shown for illustration purposes}). (right) The silhouette curves, marked in pink, are re-projected in 3D. 
\AT{why in the image in the left it's only the image-space silhouette with a pink boundary, while on the right you can also find the silhouette in the interior?}
\Anastasia{As I mentioned, for illustration purposes, to show to the outline is always outside of the model. Probably it is a bad idea, I should just display the same outline as on the right, because that is what I actually compute.}
} 
\label{fig:silhouette}
\end{figure}