\begin{figure}[t!]
\centering
\begin{overpic} 
[width=\linewidth]
% [width=\linewidth,grid,tics=10]
{\currfiledir/item.pdf}
\put(9,56){{\small $E_{2D}$ }}
\put(22,57.15){{\tiny \cite{tagliasacchi2015robust}}}
\put(22,55.15){{\tiny [Proposed Method]}}
%
\put(44.5,56){{\small $E_{3D}$ }}
\put(57,57.15){{\tiny \cite{tagliasacchi2015robust}}}
\put(57,55.15){{\tiny [Proposed Method]}}
% 
\put(11,2){{\small \emph{tayl1} }}
\put(22,2){{\small \emph{srid1} }}
\put(32,2){{\small \emph{srid2} }}
\put(43,2){{\small \emph{srid3} }}
\put(54,2){{\small \emph{srid4} }}
\put(64.5,2){{\small \emph{shar1} }}
\put(75,2){{\small \emph{shar2} }}
\put(86,2){{\small \emph{shar2} }}
% 
\end{overpic}
\caption{
% 
% 
Aggregated 2D/3D tracking performance metrics of the proposed method compared to \protect\cite{tagliasacchi2015robust}. 
% 
In the additional materials we report error plots through time for the aggregated data above.
% Our calibrated convolution model often helps the generative  technique from incurring in loss-of-tracking, these are visualized are spike in the errors in the
% 
% 
}
\label{fig:barchart}
\end{figure}