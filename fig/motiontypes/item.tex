\begin{figure}[t!]
\centering
\begin{overpic} 
[width=\linewidth]
% [width=\linewidth,grid,tics=10]
{\currfiledir/item.pdf}
% \put(40,25){\todo{\Large DRAFT}}
% \put(0,-3){\todo{\currfiledir}}
\end{overpic}
\caption{
% 
Our dataset contains a number of wide range of motions that spans the entire literature on hand tracking. We identify three main axes of complexity by analyzing recent hand-tracking papers, and produce the \handyseq{teaser} sequence to sample this space. 
% 
For each motion type we show a pair of images 150ms apart to illustrate the motion.
\Anastasia{The larger coordinate at each axis the more difficult is the motion to track with monocular depth sensor}
% 
}
\label{fig:motiontypes}
\end{figure}