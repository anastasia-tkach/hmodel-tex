\begin{figure}[t!]
\centering
\begin{overpic} 
[width=\linewidth]
% [width=\linewidth,grid,tics=10]
{\currfiledir/item.pdf}
% \put(40,25){\todo{\Large DRAFT}}
% \put(0,-3){\todo{\currfiledir}}
\end{overpic}
\caption{
% 
Our dataset contains a wide range of motions that cover sequences shown in previous work on hand tracking. We identify three main axes of complexity by analyzing recent hand-tracking papers, and produce the \handyseq{teaser} sequence that samples this space; see \Video{00:00}. 
For each motion depicted in the graph, the distance to the origin gives an indication of the level of difficulty for tracking.
% 
}
\label{fig:motiontypes}
\end{figure}