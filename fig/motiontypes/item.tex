\begin{figure}[t]
\centering
\begin{overpic} 
[width=\linewidth]
% [width=\linewidth,grid,tics=10]
{\currfiledir/item.pdf}
% \put(40,25){\todo{\Large DRAFT}}
% \put(0,-3){\todo{\currfiledir}}
\end{overpic}
\caption{
% 
Our dataset contains a number of wide range of motions that spans the entire literature on hand tracking. We identify three main axes of variations by analyzing recent hand-tracking papers, and produce a dataset that samples this space. \TODO{export the same image, but **without** text, also fill the entire width so to make the models bigger. Also you color code, but then you use two axes with the **same** color? Then what's the point of color-coding? Also, you need to add numbers to each of them! (needed for a figure later)}
% 
}
\label{fig:motiontypes}
\end{figure}