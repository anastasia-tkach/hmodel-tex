\documentclass[tog]{acmsiggraph}
\usepackage{color}
\usepackage{overpic}
\usepackage{amssymb}
\usepackage{graphicx}
\usepackage{epstopdf}
\usepackage{amsmath}
\usepackage{graphicx}
\usepackage{amsfonts}
\usepackage{enumitem}
\usepackage{multirow}
\usepackage{algpseudocode}
\usepackage{algorithm,algorithmicx}
\usepackage[dvipsnames]{xcolor}


% \usepackage{cite}


%% 
%% Insert symbols on figures
%%
\usepackage{overpic}
\usepackage{currfile} % \currfiledir
\usepackage{wrapfig}
\usepackage{graphicx}
\usepackage{caption} % \caption*

%% 
%% More compact paragraphs
%%
% \usepackage{stringstrings} %< TitleCase
\renewcommand{\paragraph}[1]{\textbf{#1.}}

%% 
%% Bibliography
%% 
% \usepackage[backend=biber,sorting=ydnt,style=alphabetic,citestyle=alphabetic,maxbibnames=99]{biblatex}

%%
%% text layout
%%
% Insert whitespace at the end of a paragraph
\setlength{\parskip}{.5\baselineskip}%
% Don't intent by default on new paragraph
\setlength{\parindent}{0pt}%

%% 
%% To define colors
%% 
\usepackage{color}
\definecolor{lightgray}{rgb}{0.64, 0.64, 0.64}
\definecolor{mygray}{gray}{0.6}

%% 
%% Portions that {need major work, comments}
%% 
\newenvironment{draft}{\color{red}}{\color{black}}
\newcommand{\INSTRUCTIONS}[1]{{\textcolor{blue}{[INSTRUCTIONS] #1}}}
% \newcommand{\copypaste}[1]{{\textcolor{blue}{#1}}}
\newcommand{\copypaste}[1]{{{#1}}}
\newcommand{\OK}{{\textcolor{green}{OK!}}}
\newcommand{\todo}[1]{{\textcolor{red}{#1}}}
\newcommand{\TODO}[1]{{\textcolor{red}{[TODO: #1]}}}
\newcommand{\ignore}[1]{}
\newcommand{\ATFINISH}{\todo{{----------- ANDREA - FINISHED EDITING HERE -----------}}}

%% 
%% Inlined comments
%% 
\newcommand{\AT}[1]{{\textcolor{blue}{[AND: #1]}}}
\newcommand{\Anastasia}[1]{{\textcolor{PineGreen}{[ANA: #1]}}}

%% 
%% Performs the following type of transformation:
%%   \Fig{figname} => Fig.~\ref{fig:figname}
\newcommand{\Fig}[1]{Fig.~\ref{fig:#1}}
\newcommand{\Figure}[1]{Figure~\ref{fig:#1}}
\newcommand{\Eq}[1]{Eq.~\ref{eq:#1}}
\newcommand{\Sec}[1]{Sec.~\ref{sec:#1}}
\newcommand{\Section}[1]{Section~\ref{sec:#1}}
\newcommand{\Appendix}[1]{Appendix~\ref{app:#1}}
\title{Convolution surfaces model  for hand tracking}
\author{}
\pdfauthor{Secret}

\teaser{
    \includegraphics[height=1.5in]{fig/sampleteaser.pdf}
    \caption{Spring Training 2009, Peoria, AZ.}
}

\begin{document}
\maketitle

\begin{abstract}
%--- Background
Modern systems for real-time tracking rely on a combination of discriminative and generative approaches to robustly recover the hand's pose. Generative approaches require the specification of the geometric model, where the fitness of this model to the user heavily affects tracking precision.
%--- Core point
In this paper, we propose to use a special instance of convolution surfaces as a novel geometric representation for real-time hand tracking. 
%--- Model Adaptation
We first derive an optimization to non-rigidly deform the template to fit the user data in a number of poses, thus producing a 3D model that jointly captures the user's static and dynamic geometry.

\Anastasia{Initial transformations of the fingers are as important for tracking as precise geometry. We find both geometry and transformations.}

%--- Performance
While the convolution model fits the model tightly, allowing high-precision registration, the limited number of primitives in the model allows us to retain excellent tracking performance. We confirm this by embedding our models in an open source real-time tracking algorithm and obtaining a tracker steadily running at 60 frames per second.

\Anastasia{Moreover, we introduce the fist model type that represents all the degrees of freedom of a real hand without introducing non-existing degrees of freedom. The model type allows to natively specify whether the current part is rigid, articulated or elastic.}

\Anastasia{The increased tracking fidelity of the model has allowed us NOT TO USE RE-INITIALIZATION AT ALL. The absence of re-initialization component in our system is not a drawback of our approach, there is nothing that prohibits to add it. Our intension is to push pure tracking as far as possible and to demonstrate its power.}

%--- Why should I believe you?
We demonstrate the effectiveness of our solution by qualitatively and quantitively evaluating tracking precision across a number of users, and on a variety of complex motions.
%--- Data release
To enable further research in the area of high-precision tracking, we disclose our datasets altogether with the corresponding quantitative evaluation metrics.
\end{abstract}
% Problem Statement: What's the problem you want to solve?
% 
% Motivation: Why is this an interesting problem? Who cares about it? Why now? Why is it appropriate for the conference audience?
% 
% Research Gap, Novelty: Why is new research required? Why can the problem not be solved with existing methods? How does the proposed solution differ from and/or improve upon existing work?
% 
% Technical Contribution: What's the key technical idea to solve the problem? Why is it beautiful?
% 
% Applications / Future Work: What will your solution enable? How does it project into the future? How will it inspire future work?



% NOTE: the paragraph names can be removed later on
\section{Introduction}

%--- Why hand-tracking is fundamental (something abour AR/VR?)
% Hand tracking is a process of accurately reconstructing shape and articulation of human hands. It is a crucial component of natural human-computer interfaces and animation of humanoid avatars.
\TODO{In our world, our hands are our main mean of interaction, and with the advent of augmented and virtual reality experiences, there is a compelling need ...}
%--- Camera-based tracking & Physics
Accurate real-time hand tracking is therefore a fundamental technical challenge \todo{In AR and VR applications a 3D hand model can properly interact with 3D objects, establish a realistic contact and disappear behind them. Given that, the degree of immersion into virtual reality depends on whether a user sees own realistic hands (find a study mentioned by Leap Motion).} \todo{Lorem ipsum dolor sit amet, consectetur adipisicing elit, sed do eiusmod tempor incididunt ut labore et dolore magna aliqua. Ut enim ad minim veniam, quis nostrud exercitation ullamco laboris nisi ut aliquip ex ea commodo consequat. Lorem ipsum dolor sit amet, consectetur adipisicing elit, sed do eiusmod tempor incididunt ut labore et dolore magna aliqua. Ut enim ad minim veniam, quis nostrud exercitation ullamco laboris nisi ut aliquip ex ea commodo consequat.}

%--- Combination of model+appearance
\paragraph{Tracker: discriminative v.s. generative.}
Modern systems for real-time tracking \cite{sridhar2015fast,sharp2015accurate},  rely on a combination of \emph{discriminative} approaches like \cite{oberweger2015feedback}, and \emph{generative} approaches such as \cite{tagliasacchi2015robust}. The per-frame re-initialization of discriminative methods prevents error propagation by offering a continous recovery from tracking failure. As these discriminative models are learnt from data, they typically only estimate a coarse and jittery pose. Therefore, generative models are used to refine the estimate by aligning a geometric template of the user hand to the measured point cloud \AT{temporal?}. It is not surprising that the quality of the template directly affects the quality of pose refinement; see \Figure{teaser}. 
\TODO{mention that no need to re-init at high FPS?}.
% 
Therefore, the main goal of this paper is to explore novel tracking templates that strike an optimal \emph{balance} between accuracy and performance, that is, a model that is able to more accurately capture the user's geometry, while retaining the ability to answer registration queries in close form with very high efficiency.

\paragraph{Surface and volumetric templates.}
In modern digital production representing objects by a piecewise linear meshing of their surface (i.e. triangular or quad meshes) is the de-facto standard. However, unlike volumetric models~\cite{bloomenthal1997book}, surface representations cannot efficiently answer queries such as the distance from the point to the object's boundary, or whether a point lies inside/outside the model~\cite[Ch.1]{botsch2010book}. In tracking applications these queries play a fundamental role, as the optimization attempts to find configurations where the average \emph{distance} from model to data is minimized. Similarly, a tracker should prevent the model from assuming implausible configurations, for example by preventing self-intersections as measured by inside/outside predicates. For all these reasons, volumetric models appear optimal for registration applications; indeed, recent compelling results in joint rigid registration and reconstruction~\cite{newcombe2011kinfu} as well as its recent non-rigid variant~\cite{newcombe2015dynfusion} leverage volumetric models. One important observations is that such techniques assume the frame-rate is high compared to camera and user motion, a condition that is surely not realizable in our setting. \AT{something about the fact that deforming a surface is easier than deforming a volume?} To tackle this challenge, in this paper we propose to employ a \emph{hybrid} model for tracking that inherits the advantages of surface and volumetric representations. 

\newpage
\begin{figure}[t!]
\begin{overpic} 
[width=\linewidth]
% [width=\linewidth,grid,tics=5]
{fig/convsurf/item.pdf}
\put(20,51){\small{$\mathcal{M}$}}
\put(70,51){\small{$\mathcal{M}$}}
\end{overpic}
\vspace{-.3in}
\caption{
% 
% 
The mesh {\small$\mathcal{M}$} explicitly identifies sphere positions and controls the union convolution operator to generate an implicit function. The zero-crossing of this implicit function describes the convex-hull of our spheres. \AT{redo this figure as I stole if from [ryoichi-sig13]!! \TODO{add the vertices names $m_1$ and $m_2$}}
% 
% \ignore{~\cite{ryoichi_sig13}}.
}
\label{fig:convsurf}
\end{figure}

\paragraph{Hybrid tracking model.}
The model we propose in this paper is a variant of a convolution surface~\cite{bloomenthal1991convolution}, and its fundamental building block is illustrated in \Figure{convsurf}. Such a construct is nothing but the zero iso-surface of the scalar function:
\begin{equation}
\phi(\mathbf{x}) = \min \int_{\mathbf{c} \in \mathcal{M}} \mathcal{B}_{\mathbf{c}, r(\mathbf{c})}(\mathbf{x}) \: d\mathbf{c},
\label{eq:convsurf}
\end{equation}
where $\mathcal{M}$ is a skeletal control mesh (a segment or a triangle with respect to the simple example in \Figure{convsurf}), and $\mathcal{B}$ is the implicit function of a sphere parameterized by its center $\mathbf{c}$ and radii $r$:
\begin{equation}
\mathcal{B}_{\mathbf{c}, r(\mathbf{c})}(\mathbf{x}) = \|\mathbf{x}-\mathbf{c}\|^2 - r(\mathbf{c})^2.
\end{equation}
The spheres centers $\mathbf{c}$ span the skeleton $\mathcal{M}$, while the radii is a function of the position $\mathbf{c}$ within an element, and more specifically, it is linearly interpolated from values specified on the skeletal mesh vertices $r(m_*)$. This is indeed a \emph{hybrid} model, as \Eq{convsurf} defines an implicit surface $\mathcal{S} = \{\mathbf{x} \in \mathbb{R}^n | \phi(\mathbf{x})=0 \}$, while the underlying skeleton $\mathcal{M}$ is an explicit representation (a piecewise-parameterized \todo{surface}). We generalize this basic construct to devise a model suitable to represent a human hand; see \Figure{topology}.
\begin{figure}[t!]
\centering
\begin{overpic} 
[width=\linewidth]
% [width=\linewidth,grid,tics=10]
{fig/topology/item.pdf}
%{fig/topology/topology}
% \put(10,10){\todo{fig:topology}}
\end{overpic}
\caption{
% 
% 
(left) The skeleton $\skeleton$ parametrizes the convolution surface providing a radii value property on vertices. 
In our template, articulated components are shown in {\color{darkgreen} dark green} while flexible components in {\color{purple}purple}.
% 
(right) Calibration instantiates our template by adjusting the skeletal vertex positions and radii. 
% 
% 
}
\label{fig:topology}
\end{figure}

\paragraph{Tracking and calibration with convolution models.}
Our novel tracking model has two significant advantages. (1) First of all, distance queries to $\surface$ can be executed by measuring the distance to the skeletal structure $\skeleton$. The number of elements in $\skeleton$ is significantly smaller (\todo{25} in our model) than the number of polygons in a typical triangular mesh representation of $\surface$~\cite{thiery2013sphere}. Therefore, not only our distance queries can be executed with great efficiently as we can evaluate distance queries to $\skeleton$ in brute force but this leads to an algorithm that is trivially parallelizable and that executes with a fixed framerate. (2) Our hand model parameterization is also compact, as we can generate a family of models by simply adjusting \emph{positions} and \emph{radii} of the control skeleton vertices $m_* \in \skeleton$. A direct approach to building a calibrated model is to let the artist manually adjust these quantities; this process is known as  \emph{ZSphere~{\textcopyright}} modeling in the popular \emph{ZBrush~\textcopyright} 3D modeling software.

\TODO{--- STOPPED HERE -- BACK TOMORROW ---}

\begin{figure}[b]
\centering
\begin{overpic} 
[width=\linewidth]
% [width=\linewidth,grid,tics=10]
{fig/coarsemodel/item.pdf}
\put(10,10){{\Large TODO}}
\put(40,10){{\Large TODO}}
\put(75,10){{\Large TODO}}
\end{overpic}
\vspace{-.25in}
\caption{(left) Tracking when the model from \protect\cite{tagliasacchi2015robust} is used without proper coarse scale calibration. (middle) A roughly manually calibrated model can help increasing the fitting fidelity, but tuning becomes increasingly difficult with more degrees of freedom. (right) The proposed automatically calibrated model.}
% \caption{(left) coarse hand model from \protect\cite{tagliasacchi2015robust} \todo{(right) hand model animated with linear blend skinning from \protect\cite{sharp2015accurate}}}
\label{fig:coarsemodel}
\end{figure}
\newpage

% While the integral in \Eq{convsurf} might be difficult to evaluate for generic skeletal structures $\mathcal{M}$,

% \TODO{why is it necessary, plus previous works}
\paragraph{Model calibration.}
As generative models perform tracking by fitting a geometric model to what is measured by the sensor, our template should be able to represent well the observed data. However, even in modern trackers, the discrepancy between the optimal model pose given the data and the true hand pose can be significant. For example, the sources of discrepancy in \Figure{coarsemodel} include the incorrect length of fingers and the lack of the \todo{[pinky-ring]} degree of freedom. \todo{In the literature, the creation of user-specific models for tracking is referred to as \emph{calibration}.}

\paragraph{LBS Looks Crappy}
\todo{The hand skinning quality is obviously important for digital avatars applications. The simple skinning approaches like linear blend skinning may generate implausible results.}

\begin{figure}[t!]
\centering
\begin{overpic} 
[width=\linewidth]
% [width=\linewidth,grid,tics=10]
{fig/handmodels/hand_model_representations.png}
% \put(10,10){\todo{\Large Overlay Text}}
\end{overpic}
\caption{Hand model representations from works of (a) Qian et. al., (b)  Oikonomidis et. al., (c) Melax et. al.  and (d) Sharp et. al. \AT{inline the names}}
\label{fig:onecol}
\end{figure}

% \begin{figure}[h!]
% \centering
% \hspace{-2em}
% \includegraphics[width=0.5\textwidth]{fig/hand_model_representations}
% \caption {Hand model representations from works of (a) Qian et. al., (b)  Oikonomidis et. al., (c) Melax et. al.  and (d) Sharp et. al}
% \label{fig:hand_model_representations}
% \end{figure}

\begin{figure}[t!]
\centering
\begin{overpic} 
[width=\linewidth]
% [width=\linewidth,grid,tics=10]
{fig/topology/item.pdf}
%{fig/topology/topology}
% \put(10,10){\todo{fig:topology}}
\end{overpic}
\caption{
% 
% 
(left) The skeleton $\skeleton$ parametrizes the convolution surface providing a radii value property on vertices. 
In our template, articulated components are shown in {\color{darkgreen} dark green} while flexible components in {\color{purple}purple}.
% 
(right) Calibration instantiates our template by adjusting the skeletal vertex positions and radii. 
% 
% 
}
\label{fig:topology}
\end{figure}


\begin{table}[!ht] 
	\centering
	\begin{tabular}{|p{2.5cm}|p{2.5cm}|p{2.5cm}|}
	\hline
 	& Hand pose  & Hand shape  \\
	\hline
	Triangular mesh with embedded skeleton, \cite{taylor2014user} & Vertices and bones positions & Vertices and bones positions	 \\
	\hline
	Cylinder model, \cite{tagliasacchi2015robust} & Cylinders size and transformations & Cylinders transformations	 \\
	\hline
	Convolution surfaces model & Positions and radii of control points & Positions of control points \\
	\hline
	\end{tabular}
	\vspace{1em}
	\caption{Comparison of different hand model representations}
	\label{table:representation_dependent_components}
\end{table}

\subsection{Convolution surfaces for model fitting}
The spheres and mixed cylinders/spheres hand model representations (Figure \ref{fig:hand_model_representations} a, b) are ubiquitous in hand tracking, because they are well suited for tracking tack per se (see next) and can be quickly to created manually. If a small number of  building blocks is used, the precision of the model is low, especially in the palm region. A higher precision can be obtained by increasing the number of primitives, which defeats the purpose of model simplicity. Convolution surfaces representation gives higher precision for the same number of building blocks. \textcolor{mygray}{Add experimental or theoretical support for convolution surfaces.}
The triangles mesh representation can approximate the hand to a high precision. However, it is not trivial to specify the model parts that are rigid and should be kept the same shape between the poses. This results in overfitting to local skin deformations. 

\subsection{Convolution surfaces for hand tracking}
For model based tracking the main operation is to find the closest point on the model for a given data point. This operation is can be done in closed for each rigid segment with spheres/cylinder and convolution surfaces model representation. 
For a triangular mesh this operation has complexity linear in number of triangles. Thus, it is necessary to simplifying assumptions and/or more complex optimization to allow the hand tracking system to run in real time. (Look how different systems deal with this problem). Moreover, the triangular mesh has (much) more degrees of freedom than the underlying problem. Without additional regularization, rigid parts of the hand model can deform to fit the data and the individual vertices can shift to fit the sensor noise.

\subsection{Convolution surfaces for hand skinning}
The Linear Blend Skinning approach used to pose the triangular mesh model in previous works (\cite{sharp2015accurate}, \cite{schroder2013analysis} ) creates artifacts, the fingers looks like made from rubber. The spheres/cylinders model is not suitable for realistic animation, therefore a re-targeting step to a template mesh is required. Retargeting does not only demand additional effort, but also brings additional imprecision. The state of the art approaches in hand skinning are implicit surfaces-based (\cite{vaillant2013implicit},  \cite{vaillant2014robust} ).  A convolution surfaces model serves a ready to use input for such an approach.

\subsection{Contributions}

\subsubsection*{1. Suggested a hand model representation well suited for user-specific adaption, tracking and animation}

TODO: 

\begin{itemize}
\item Modeling. Implement initialization for the model fitting by running Htrack and putting the hand in a similar pose. 
\item Skinning. Display the texture on the model. For that a parametrization is required. Potentially create ``bumps map'' to apply the errors from model fitting stage. 
\end{itemize}

\subsubsection*{1a. Proposed a single optimization framework for modeling and tracking}

TODO:

\begin{itemize}
\item Run a couple of iteration of ARAP for palm after IK optimization just with data term.
\end{itemize}

\subsubsection*{1b. Developed a tracking model that represents all the degrees of freedom of human hand without creating inexistent degrees. (Not guaranteed by the other approaches, look how model adaptation paper finds the degrees of freedom)}

\subsubsection*{Improved accuracy of state-of-art hand tracking system}

TODO: 

\begin{itemize}
\item Add outline term, time interpolation of initial joint angles, downsampling of the sensor data, fingertips term.
\end{itemize}
\subsection{Contributions}

\begin{itemize}

\item Developed an approach for approximation a model with convolution surface, given a skeleton topology.
\item Formulated position-based inverse kinematics algorithm for hand tracking. The position based approach does not require walking through the hierarchical chain of joint, thus saving the computational power. Also, position-based inverse kinematics does not involve linear approximations or rotations, thus making the optimization more stable.
\item Suggested an automatic approach for field functions construction in implicit skinning.  \textcolor{mygray}{Replaced newton iteration for vertex projection by a closed form solution.}

\end{itemize}


%%%%%%%%%%%%%%%%%%%%%%%%%%%%%%%%%%%%%%%%%%%%%%%%%%%%%%%%%%%%%%%%%%%%%%%%

\section{Related Literature}

\begin{itemize}

\item Albrecht et al. \cite{albrecht2003construction} developed an approach for creating an anatomically realistic hand model that includes bones and muscles structure. Their approach requires several prerequisites including plaster cast of a human hand and laser scanner for manually creating a physically realistic hand template. Given user-defined correspondences between 3D feature points and the hand image, a specific hand model is created by deforming a generic hand model. 

\item Rhee et. al. \cite{rhee2006human} use a single image of a hand at rest pose to infer joint hand joint locations from skin creases. Given the skeleton obtained at the previous step and the hand contour from the image, they deform a template hand mesh to fit this data. 

\item Straka et al. \cite{straka2012simultaneous} also fit the template mesh with attached skeleton to 3D data. The model is deformed to explain the data while keeping the vertices attached to their corresponding bones. It is on clear whether the approach whether be able to handle a hand motion sequence, since the results are demonstrated on a full body model.

\item Taylor et. al. \cite{taylor2014user} generate a user-specific hand model from an RGBD video sequence. The model is represented as a triangular mesh with an embedded skeleton. In each frame the hand pose is initialized using an appearance-based tracking algorithm. The hand model parameters are found by solving a single optimization problem formulated for the entire video sequence which also finds hand pose in each frame. 

\item Khamis et. al.  \cite{khamis12learning} fit a hand model for a specific user by finding its shape coordinates in the basis of mesh matrices and bones locations. As in the approach by Taylor et. al, they optimize simultaneously for pose and shape parameters in all the frames of an RGBD sequence across all the subjects. Requires large number of subjects as a regularization for excessive degrees of freedom. 

The results generated by the approaches listed above could be used as an input for our system to create a hand model representation adapted for efficient tracking and animation.


\end{itemize}
\input{sec/modeling.tex}
\section{Tracking}

\paragraph{Element Projection}
\TODO{Anastasia, please add and describe \Figure{corresp}, that is how do we compute correspondences to a single convolution element}.

\paragraph{Element-Wise Correspondences}
Our correspondence search leverages the structure of \Eq{convsurf}, by decomposing the surface in several elementary convolution elements $\element$ (\todo{a decomposition of the shape in pill and wedge primitives}) and their associated implicit functions $\implicit_e$. Given a point $\point$ in space, the implicit function of the whole surface can be written by evaluating the expression:
\begin{equation}
\implicit_\surface(\point) = \argmin_{e=1 \dots E} \implicit_e(\point)
\label{eq:piecewise}
\end{equation}
Therefore, given a query point $\point$, we can first compute the closest-points $\footpoint_e = \proj_{\element}(\point)$ to each element independently; within this set, the closest-point projection to the full model $\footpoint = \proj_\surface(\point)$ is the one with the smallest associated implicit function value $\implicit_e(\point)$.
\TODO{should we mention how long this takes?}

\begin{figure}[t]
\centering
\begin{overpic} 
[width=\linewidth]
% [width=\linewidth,grid,tics=10]
{fig/corresp/item.pdf}
% \put(10,10){\todo{\Large Overlay Text}}
\end{overpic}
\caption{
% 
% 
The computation of closest point correspondences on pill (left) and wedge (right) convolution elements can be performed by tracing a ray along the normal of the circles (resp. sphere) tangent line (resp. plane). \TODO{how to subscript in illustrator?} \MP{I usually prefer overpic and use latex. It's a bit painful, but usually looks much nicer}
%
%
}
\label{fig:corresp}
\end{figure}
\paragraph{Frontal Correspondences}
In monocular acquisition, an oracle registration algorithm aligns the portion of the model that is \emph{visible} from sensor viewpoint to the available data. Therefore, when computing ICP's closest-point correspondences, only the portion of the model currently visible by the camera should be considered~\cite{tagliasacchi2015robust}. Given the camera direction $\camdir$, we can test whether the retrieved footpoint $\footpoint$ is back-facing by testing the sign of $\camdir \cdot \mathcal{N}_\surface(\footpoint)$, where the second term is the object's normal at $\footpoint$. As illustrated in \Figure{?}, whenever this test fails, there are additional candidates for closest point that must be checked . The first with respect to the silhouette of the model, 

The \emph{silhouette} of an object is a (3D) curve separating front from back-facing portions of a shape. If we restrict correspondence search to the frontal portion of the model, naturally many correspondences will be mapped to such curve. 

\FINISH

% In convolution surfaces correspondence computation takes place in four steps:
% \todo{If our convolution model would be composed of a single skeletal element, the computation of correspondences would be as visualized in~\Figure{corresp}.}

% \begin{figure}[t!]
\centering
\begin{overpic} 
[width=\linewidth]
{fig/silhouette/item.pdf}
\end{overpic}
\caption{
% 
% 
(left) The silhouette of the model computed by projecting the model in the camera plane (\todo{fingers outline is computed separately, here an entire model outline is shown for illustration purposes}). (right) The silhouette curves, marked in pink, are re-projected in 3D. \AT{this figure needs to be re-done, but let's deal with it later. I will give you precise directions}
% 
% 
} 
\label{fig:silhouette}
\end{figure}

\begin{algorithm}
\caption{Correspondences computation}
\begin{algorithmic}[1]
    \For {$\text{each } p$}
    	 \State \text{compute model projection } $q_m$
    	 \State \text{replace or discard if $q_m$ is back-facing}
    	 \State \text{compute outline projection } $q_o$
         \State $q=(\|{p - q_m}\|_2^2 < \|{p - q_o}\|_2) ? q_m : q_o$
    \EndFor
\end{algorithmic}
\label{alg:correspondences}
\end{algorithm}

% \textbf{Computing model projection.}
% taking the minimum helps to get the projection on the model surface when the data point $p$ is inside of the model \AT{how? I am not sure I understand, please quickly sketch a figure!)}.

\textbf{Discarding or replacing back-facing projections.}
\begin{itemize}
	\item Convolution segment: the closest front-facing point is on the model outline, thus set the current point $q_m$ to $\infty$.
	\item Convolution triangle: the closest front-facing point is either on the model outline or on a front-facing face of the convolution triangle, thus replace $q_m$ by the closest front-facing face projection.
\end{itemize}

\textbf{Computing outline.}
For this computation we assume that projection is orthographic and that camera direction coincides with axis $Z$. We shift all the model spheres to have zero coordinate at axis $Z$ and compute an outline of the cross-section of the model with the $XY$ plane. This outline is computed by finding the upper left point of the cross-section and traversing the graph of line and circle segments in counters-clockwise direction (see Figure \ref{fig:silhcorr}). To obtain the 3D outline we shift the model spheres with attached 2D outline back to their original positions.
\AT{What's the difference between the silhouette here and the one computed by raytracing/rasterizing the model on the GPU?}




\textbf{Computing outline projection.}
The model outline is represented as a sequence of line and circle segments. To compute an outline projection $q_o$ we compute a projection on each element of the outline and select the closest to the data point $p$.

\begin{figure}[t]
\centering
\begin{overpic} 
[width=\linewidth]
% [width=\linewidth,grid,tics=10]
{fig/visibility/item.pdf}
% \put(10,10){\todo{\Large Overlay Text}}
\end{overpic}
\caption{In a monocular RGBD sensor only the front-facing part of the model should be registered to the data. Here the camera is observing (left to right) two 2D convolution elements and the visible parts of the model are color-coded. Correspondences whose normal point away from the camera ought to be discarded, and replaced by the closest amongst silhouette correspondences or front-facing portions of wedges.}
\label{fig:visibility}
\end{figure}

%%%%%%%%%%%%%%%%%%%%%%%%%%%%%%%%%%%%%%%%%%%%%%%%%%%%%%%%%%%%%%%%%%%%%%%%%
\subsection{Joint limits}


\begin{figure}[h!] 
	\centering
	\hspace{0em}
	\includegraphics[trim = 0mm 70mm 0mm 0mm, width=0.5\textwidth]{fig/joint_limits}
	\caption{The computation of the limiting rotation}
	\label{fig:limiting_rotation}
\end{figure}

Joint limits energy is formulated similarly to the ARAP energy. The current edge rotation $R_e$ is replaced by limiting rotation $R_L$ in case if the joint limits are violated (if $I_a = 1$). Otherwise, the joint limits energy is not active ($I_a = 0$).

\begin{equation}
	E_{L} = \omega_3 \sum_{a \in E} I_a \| a(c^t) - R_L a(c^0)\|_2^2,
\end{equation}

The procedure of finding the limiting rotation $R_L$ is presented on Figure \ref{fig:limiting_rotation}. In the equation \ref{eq:limiting_rotation}, $R_a$ and $R_b$ are the rotations w.r.t the rest pose of the current edge $a$ and its parent $b$. $R_{b\rightarrow a}$ is the local rotation of $a$ w.r.t $b$. $\tilde{R}_{b\rightarrow a} $ is the local rotation clumped to the limit joint angle value.
 
\begin{equation}
	R_L = R_b \tilde{R}_{b\rightarrow a} R^T_{b\rightarrow a} R^T_b R_a \label{eq:limiting_rotation}
\end{equation}

Given the Euler decomposition of the local rotation  $R_{b\rightarrow a} = R_x(\theta_x) R_z(\theta_z) R_y(\theta_y)$, the limiting rotation is computed by clamping the joint angles along the constrained axis.  $\tilde{R}_{b\rightarrow a} = R_x(\tilde{\theta}_x) R_z(\tilde{\theta}_z) R_y(\tilde{\theta}_y)$. Note the order of Euler decomposition. In our settings, $x$ is axis of flexion, $z$ is axis of abduction and $y$ is axis of twist. It is important to find the twist component last. In the expression 

\begin{equation}
\tilde{R}_{b\rightarrow a} R^T_{b\rightarrow a} =  R_x(\tilde{\theta}_x) R_z(\tilde{\theta}_z) R^T_z(\theta_z) R^T_x(\theta_x)
\end{equation}

the twist component is discarded, because it does not make a difference.

%%%%%%%%%%%%%%%%%%%%%%%%%%%%%%%%%%%%%%%%%%%%%%%%%%%%%%%%%%%%%%%%%%%%%%%%%

\section{Results}

\subsection{Modeling}

\begin{figure}[h!] 
	\centering
	\includegraphics[width=0.5\textwidth]{figures/modeling}
	\caption{Fitting convolution surfaces hand model to several poses of the same hand.}
	\label{fig:modeling}
\end{figure}

\subsection{Tracking}

In this paper instead of a standard inverse kinematics approach for aligning the model with data we use \textit{"As Rigid As Possible"} approach. The hand model is parametrized with the locations of the vertices of the hand skeleton $c = {c_1, c_2, ... c_N}$.

\begin{equation}
	\min_{c} E_{ICP} + E_{ARAP} \label{eq:tracking_energy}
\end{equation}

The first energy $E_{ICP}$ models a 3D geometric registration in the spirit of ICP as

\begin{equation}
	E_{ICP} = \omega_1 \sum_{p \in P} \| p - \Pi(p, c)\|_2
\end{equation}

where $P$ is the set of data points.

The second energy $E_{ARAP}$ is needed for shape preservation. Denote locations of hand skeleton vertices at rest pose as $c^0$ and in iteration $t$ as $c^t$. Denote the set of all edges of hand skeleton as $E$.

\begin{equation}
	E_{ARAP} = \omega_2 \sum_{e \in E} \| e(c^t) - R_e e(c^0)\|_2^2,
\end{equation}

where $R_e$ is the optimal rotation to bring the rest pose edge $e^0$ to the current position $e^t$. Unless we what some set of edges to rotate as a solid body, the rotation $R_e$ can be expressed in the closed form, such that $R_e e(c^0)$ is collinear to $e(c^t)$. 
Each optimization iteration consists of two alternating steps: first we find the projections of the data points to the model surface $\Pi(p, c)$ and the optimal rotations $  R_e  $. Then we make one step of Levenberg-Marquardt iteration for energy (\ref{eq:tracking_energy}).

\subsubsection{Trying to make several optimization steps while keeping the same correspondences}

I am not sure how to implement what you suggested yesterday. If the data-model correspondences are kept fixed, after the first update of the parameters the model points are not in the model surface anymore. So, the optimization does not make much sense (see Figure \ref{fig:fixed_correspondences}). If I do such optimization, the model just floats away from the data.
I also tried doing several iterations of $E_{ARAP}$ only, The initial length of the segment are, of course, restored. However, the model does not take the data into account during these iterations, so the optimization takes more time to converge.

\begin{figure}[h!] 
	\centering
	\includegraphics[width=0.5\textwidth]{figures/fixed_correspondences}
	\caption{Updating the centers locations while keeping the data-model correspondences fixed. The model is black, the data is pink, the corresponding points are connected in blue.}
	\label{fig:fixed_correspondences}
\end{figure}
 \section*{Acknowledgements}

Chris Wojtan: help rendering model in Fig. XX
Misha Kazhdan: insightful discussions
\bibliographystyle{acmsiggraph}
\nocite{*} %<< prints out everything
\bibliography{hmodel}

\end{document}
